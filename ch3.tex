\chapter{O Município de Maracanaú}

Neste segundo capítulo será feita uma analise que nos mostrará um pouco sobre o município de Maracanaú,
saberemos as principais características desse município, clima, economia, história, quem é o prefeito etc.

\section{Aspectos gerais de Maracanaú}

A partir de uma analise de dados sobre Maracanaú disponível no Instituo Brasileiro de Geografia e Estatística, 
conheceremos um pouco mais sobre este município que abriga vários trabalhadores informais. 

Maracanaú fazia parte do município de Maranguape até que em 1983 conseguiu torna-se independente. Através da Lei 
10.811 que determinou a criação deste grande município que é Maracanaú. E o significado do nome dado a este novo 
município que foi criado em 1983 quer dizer “ Lugar Onde bebem as Maracanãs” que é uma palavra de origem Tupi.
Com relação a sua extensão Maracanaú possui uma área territorial de 105,70$km^2$ e possui um clima predominante 
tropical quente sub-úmido, com temperatura média entre 26$^{\circ}C$ e 28${\circ}C$ e um relevo de tabuleiros pré-litoraneos

Maracanaú tem sua divisão político-administrativa nos distritos de Maracanaú e Pajuçara. Sua regionalização o 
situa como município que faz parte da região metropolitana de fortaleza. No que desrespeita aos aspectos 
demográficos podemos informar que a população que reside em Maracanaú é de aproximadamente duzentos mil 
habitantes, sendo que em sua maioria é formada por crianças e jovens com idade de 0 até 24 anos e dentre 
toda essa população a maioria situa-se na zona urbana e a menor parte situa-se na zona rural.E tem como 
prefeito Firmo Camurça.

Na área da saúde, Maracanaú possui um total de 53 unidades de saúde ligada ao sistema único de saúde(SUS) 
sendo dividido entre sedes públicas  que são no número de 44 e sedes privadas sendo o número de 9 unidades.
Possui dentre suas unidades ligadas ao SUS 3 hospitais Geral, 1 posto de saúde,9 clinicas especializadas 
etc. Possui um programa que em que os agentes de saúde acompanham crianças de até 4 anos de idade. 
Seus principais indicadores de saúde são: 2,45 médicos a cada mil habitantes,0,32 dentistas a cada mil
habitantes, 1,25 leitos para cada mil habitantes, 0,25 unidade de saúde para cada mil habitantes e uma 
taxa de mortalidade de 9,0 para cada mil crianças nascidas vivas.

Na Educação o número total de professores que atuam dentro do município é de 2.766 distribuídos em escolas 
federais, estaduais e municipais e particulares. Dentre essas escolas citadas os alunos que usufruem o
direito de estudar e possuírem a sua disponibilidade alguns equipamentos essenciais dentre eles bibliotecas,
salas de aula em condições de ter aula e laboratórios de informática. Os indicadores mostram a população 
alfabetizada de Maracanaú é de aproximadamente 140 mil pessoas com mais de 15 anos.

Com relação ao emprego e renda o trabalho formal compreende uma população de aproximadamente 50 mil
trabalhadores que atuam dentro do município distribuídos dentre as principais atividades: Comércio, 
Indústria de transformação, construção civil, serviços e administração pública.

A economia do município corresponde ao PIB a preços de mercado (R\$ mil) de 3.121.055 PIB per capita(R\$ 1,00) 
15.620. Agropecuária 0,12, Indústria 57,93 e serviços 41,95. A finança pública tem uma receita total de 284.731 
e despesas de 304.382.

Com relação ao principal foco da pesquisa, os trabalhadores informais segundo uma pesquisa realizada em 2010 
realizada em Maracanaú mostram que 25,01 de cada 100 habitantes trabalha informalmente com idade entre dez anos 
ou mais e estão incluídos nesses dados aprendizes ou estagiários sem remuneração.
