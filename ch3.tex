\chapter{O Município de Maracanaú}

Neste segundo capítulo será feita uma análise que nos mostrará um pouco sobre o município de Maracanaú, saberemos as principais características dele como, por exemplo, clima, economia, história, política etc. A partir de uma analise de dados sobre Maracanaú disponível no Instituo Brasileiro de Geografia e Estatística \cite{demografico2000resultados}, conheceremos um pouco mais sobre este município que abriga vários trabalhadores informais. 

Maracanaú fazia parte do município de Maranguape até 1983 quando tornou-se independente por meio da Lei 10.811 que determinou a criação de seu município. O seu nome quer dizer: ``Lugar onde bebem as Maracanãs'' que é uma palavra de origem Tupi. 

Com relação a sua extensão, Maracanaú possui uma área territorial de 105,70 $km^2$ e possui um clima predominante tropical quente sub-úmido, com temperatura média entre 26$^{\circ}C$ e 28$^{\circ}C$ e um relevo de tabuleiros pré-litorâneos.

Maracanaú tem sua divisão político-administrativa nos distritos de Maracanaú e Pajuçara. Sua regionalização o situa como município que faz parte da região metropolitana de fortaleza. No que diz respeito aos aspectos demográficos, a população que reside em Maracanaú é de aproximadamente duzentos mil habitantes, sendo que em sua maioria é formada por crianças e jovens com idades entre 0 e 24 anos. Dentre a população, a maioria mora na zona urbana e a menor parte na zona rural e seu prefeito chama-se Firmo Camurça.

Maracanaú possui um total de 53 unidades de saúde ligada ao Sistema Único de Saúde (SUS), sendo dividido entre sedes públicas (44 unidades) e sedes privadas (9 unidades). Possui dentre suas unidades ligadas ao SUS, 3 hospitais Geral, 1 posto de saúde, 9 clínicas especializadas etc. Possui um programa onde os agentes de saude acompanham o desenvolvimento de crianças de até 4 anos de idade.

Seus principais indicadores de saúde são: $2,45$ médicos a cada mil habitantes, $0,32$ dentistas a cada mil
habitantes, $1,25$ leitos para cada mil habitantes, $0,25$ unidade de saúde para cada mil habitantes e uma 
taxa de mortalidade de $9,0$ para cada mil crianças nascidas vivas.

Na Educação, o número total de professores que atuam dentro do município é de $2.766$, distribuídos em escolas federais, estaduais, municipais e particulares. Os alunos usufruem do direito de estudar e possuem a sua disponibilidade alguns equipamentos essenciais, como bibliotecas, salas de aula em boas condições, laboratórios de informática etc. Os indicadores mostram que a população alfabetizada de Maracanaú é de aproximadamente $140$ mil pessoas com mais de $15$ anos.

Com relação ao emprego e renda, o trabalho formal compreende uma população de aproximadamente $50$ mil trabalhadores que atuam dentro do município distribuídos nas principais atividades: comércio, indústria de transformação, construção civil, serviços e administração pública.

A economia do município corresponde ao PIB a preços de mercado (R\$ $1.000,00$) de $3.121.055$ PIB per capita(R\$ 1,00) $15.620$. Agropecuária $0,12$, Indústria $57,93$ e serviços $41,95$. A finança pública tem uma receita total de $284.731$ e despesas de $304.382$.

Com relação ao principal foco da pesquisa, os trabalhadores informais, segundo pesquisa realizada em Maracanaú no ano de $2010$,  mostram que $25,01$ de cada $100$ habitantes trabalha informalmente e possuem idade entre dez ou mais anos. Estão incluídos nesses dados aprendizes ou estagiários sem remuneração.