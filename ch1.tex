\chapter{Introdução}

Dentre os temas que poderiam ser estudados e aprofundados nesta monografia, escolheu-se algo extremamente importante, o trabalho. Porém, fez-se necessário delimitar o escopo deste tema tendo em mente um assunto que fosse de extrema importância no cotidiano atual. Neste contexto, foi escolhido o trabalho informal no município de Maracanaú no estado do Ceará.

A pesquisa começa quando o debate da palavra informalidade é discutido por diversos pensadores e mesmo assim ainda não conseguiu-se chegar a uma definição aceita por uma grande maioria. A partir desses autores consegue-se descobrir que o trabalho informal no Brasil basicamente é aquele realizado sem carteira assinada que não caracterize atividade de estreita sobrevivência.

Será discorrido o cotidiano de alguns trabalhadores informais através de entrevistas e uma etnografia. O que mais procurou-se foi mostrar a importância do trabalho informal para todos aqueles que estão inseridos nessa atividade, e que esta forma de trabalho não está a margem do capitalismo.

Buscou-se entender o valor atribuído ao trabalho informal para todos aqueles que estão inseridos nele, seja direta ou indiretamente. E porquê optaram por escolher trabalhar informalmente e o que consideram do trabalho informal.

Abordou-se também o conteúdo relacionado ao trabalho informal e suas precariedades. Destacou-se também situações onde, apesar de muitas adversidades, existem pessoas que conseguem se destacar neste ambiente tão hostil e competitivo. 

Pesquisou-se sobre como os grandes pensadores, dentre eles Ricardo Antunes \cite{antunes1999sentidos}; Maria Augusta Tavares \cite{augusta}; G. Noronha \cite{noronha2003informal}; Robert Castel \cite{castel1998metamorfoses} encaram o tema. Também buscou-se informações relevantes colhidas através do censo do IBGE para completar o entedimento do trabalho informal e verificar quais são os fatores que o tornam precarizado e como ele ocorre na prática.

Portanto, será mostrada ao longo da pesquisa a ligação existente entre informalidade e precarização dentro do trabalho informal, assim como seus fatores positivos e negativos, perfil de pessoas que trabalham informalmente no Maracanaú e o que mudou na vida delas que começaram a trabalhar desta maneira.
