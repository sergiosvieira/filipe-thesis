% Modelo de Trabalhos em LaTeX da UFC
%
% Na criação deste modelo foi tomada como base os modelos de monografia da UECE
% criados por Rudy Matela e Sergio Correia, sem a ajuda deles este trabalho teria sido muito mais
% difícil. Este modelo utiliza o abnTeX e um pacote (ufc.sty) para formatação
% de alguns anexos necessários da UFC (folha de rosto, CIP, epígrafe, ...).
%
% Este documento não clama possuir conformidade de 100\% com as normas de
% trabalhos da UFC. Consulte os guias oficiais.
%
% Agradeço ao Lincoln que contribui com o brasão da UFC.
% Agradeço ao Regis e a Mônica pelas contribuições com relação ao "Lorem
% ipsum"
%
% OBS: O modelo de monografias da UECE criado por Rudy Matela encontra-se
% disponível em: http://matela.com.br/pub/modelo_monografia/
%
% Autor: Diego Victor Simões de Sousa
% Data: 10/06/2011


\documentclass[pnumabnt,normaltoc,espacoumemeio,capchap]{abnt}		
\usepackage[brazil]{babel}
\usepackage[utf8]{inputenc}
\usepackage{abnt-alf}
\usepackage{graphicx}
\usepackage{ufc}
\usepackage{multicol}
\usepackage{listings}
\usepackage{eufrak}
\usepackage{subfig}
\usepackage[T1]{fontenc}
\usepackage{gensymb}
\bibliographystyle{abnt-alf}
\setcounter{secnumdepth}{3}
\setcounter{tocdepth}{3}

% Informações gerais do documento
\autor{Keveny Filipe Vieira Nunes}
\autorr{NUNES, K. F. V.}
\titulo{Trabalho Informal e Precarização no Município de Maracanaú}
\local{Redenção, Ceará}
\cidade{Redenção}
\data{2014}
\orientador{Prof. Dr. Gledson Ribeiro de Oliveira}
\coorientador{}
\codigocip{A000z}{CDD:000.0}

% Descrição para folha de rosto
\comentario{
Trabalho de conclusão de curso apresentado ao curso de Bacharelado em 
Humanidades da Universidade da Integração Internacional da Lusofonia 
Afro-Brasileira, como requisito parcial para a obtenção do título de 
Bacharel em Humanidades.
}

\comentarioaprovacao{
Trabalho de conclusão de curso apresentado ao curso de Bacharelado em 
Humanidades da Universidade da Integração Internacional da Lusofonia 
Afro-Brasileira, como requisito parcial para a obtenção do título de 
Bacharel em Humanidades.
}

% Informações institucionais
\centro{Centro de Humanidades}
\departamento{Departamento de Humanidades}
\curso{Humanidades}
\instituicao{Universidade da Integração Internacional da Lusofonia Afro-Brasileira - UNILAB}

\tipotrabalho{Monografia}
\areaconcentracao{Humanidades}
\nivel{Graduação}
\tituloacademico{Bacharel}

\dedicatoria{Dedico este trabalho de conclusão de curso a Deus e a Jesus Cristo para expressar minha imensa gratidão.}
% Epígrafe: citação e autor
% \epigrafe{``Uma citação citada deve citar o que se pretendia com a citação da citação''}
% \autorepigrafe{Zé Ninguém}

% Membros da comissão avaliadora
\bancaum{\ABNTorientadordata\\Universidade da Integração Internacional da Lusofonia Afro-Brasileira (UNILAB)\\Orientador}
\bancadois{Universidade da Integração Internacional da Lusofonia Afro-Brasileira (UNILAB)\\Co-orientador}
\bancatres{}
\bancaquatro{}

% Palavras chave
%{Informalidade}{Conceito}{Precarização}{Falta de experiência}{Desqualificação}{Ilegal}{Explotação}{Flexibilização do Trabalho}{Maracanaú}
 \pcs{Trabalho Informal}{Precarização}{Maracanaú}
 \kws{Informal Work}{Precarious}{Maracanaú}

\begin{document}

\capa
\folhaderosto
\makecippage
\termodeaprovacao

% Dedicatória (Opicional)
\makededicatoria

\pretextualchapter{Agradecimentos}
Agradeço este trabalho primeiramente a Deus e a Jesus Cristo, depois a minha mãe que me ajudou 
muito em toda minha vida, me dando todo apoio necessário para que eu pudesse me tornar algo que 
ela não conseguiu em sua vida. Apesar de todas as dificuldades encontradas ela sempre buscou 
fazer o possível e o impossível para me ajudar sem nunca medir esforços. Foram muitas adversidades 
que enfrentamos juntos e conseguimos superar o seu apoio foi fundamental em todos os momentos.

Somente através desse agradecimento quero expressar um pouco da minha felicidade e compartilhar 
com essa pessoa que merece minha profunda admiração e respeito por tudo que fez e continua fazendo por mim.

Apesar de não ser formada minha mãe sempre me incentivou a estudar e talvez se não fosse por seu 
incentivo eu não teria chegado até aqui. Apesar de faltarem palavras para expressar minha gratidão 
para minha mãe porque só eu sei o quanto ela lutou por mim e todas as coisas que ela abdicou para me ajudar.

Agradeço também todas as pessoas da minha família que me ajudaram até aqui dentre elas posso citar: 
minha mãe Lurdenia, meu avó Caetano Arruda, minha avô Maria de Lourdes, meu tio Sergio Vieira que 
para mim é um grande intelectual, minha irmã Isabelle.

Quero deixar também aqui um agradecimento especial ao meu orientador, professor e grande intelectual 
da Universidade da Integração Internacional da Lusofonia afro brasileira: Prof.Dr. Gledson Ribeiro, que 
foi de extrema importância no meu trabalho e é um profissional que eu particularmente admiro muito 
por sua capacidade e dedicação.

Agradeço também todos os meus colegas que me ajudaram dentro e fora da universidade dentre eles agradeço 
ao, Renan Bruno, Jessica Virginia, Isaac, Fladimir Andrade, Diogo Oliveira, Ícaro Oliveira, Lucas Almeida, 
Danilo Araujo, Paulo Roberto, Leandro Breno, Alef Erlange, Aldonio Silva, Jeferson Alves, Ítalo Robert, 
que de alguma forma me ajudaram a chegar até aqui e me incentivavam a terminar esta graduação.

\pagebreak

% \makeepigrafe
\begin{resumo}
Neste trabalho iremos abordar o tema do trabalho informal mais especificamente 
no município de Maracanaú, tentaremos explicar um pouco do conceito que envolve 
a palavra trabalho informal. Mostraremos as dificuldades do trabalhador 
informal, mostraremos as conquistas desses trabalhadores, iremos mostrar um 
pouco do cotidiano de uma atividade informal especifica através de uma 
etnografia. Tentaremos demonstrar a grande importância do trabalho informal para 
o município e para a vida dos trabalhadores envolvidos e tentaremos esclarecer 
algumas duvidas pertinentes ao tema. Nesta pesquisa utilizaremos principalmente 
o pensamento de Ricardo Antunes e Robert Castel. 

\palavraschave
\end{resumo}
\pagebreak

\begin{abstract}
In this paper we address the issue of informal work more specifically
in Maracanaú, we will try to explain a little concept that involves
the informal work. We will show the difficulties of informal worker and your 
achievements, also we will show a informal activity specifies through a
ethnography. We will try to address the great importance of informal work for
the municipality and for the life of the workers involved and try to clarify
some doubt about the theme. In this paper we use mainly
the thought of Ricardo Antunes and Robert Castel.

\keywords
\end{abstract}

% \listadefiguras
% \listadetabelas
\tableofcontents
% \listadesiglas % \sigla{sigla}{Descrição}
% \listadesimbolos % \simbolo{símbolo}{Descrição}

\chapter{Introdução}

Dentre os temas que poderiam ser estudados e aprofundados nesta monografia, escolheu-se algo extremamente importante, o trabalho. Porém, fez-se necessário delimitar o escopo deste tema tendo em mente um assunto que fosse de extrema importância no cotidiano atual. Neste contexto, foi escolhido o trabalho informal no município de Maracanaú no estado do Ceará.

A pesquisa começa quando o debate da palavra informalidade é discutido por diversos pensadores e mesmo assim ainda não conseguiu-se chegar a uma definição aceita por uma grande maioria. A partir desses autores consegue-se descobrir que o trabalho informal no Brasil basicamente é aquele realizado sem carteira assinada que não caracterize atividades de estreita sobrevivência.

Sera discorrido o cotidiano de alguns trabalhadores informais através de entrevistas e uma etnografia. O que mais procurou-se foi mostrar a importância do trabalho informal para todos aqueles que estão inseridos nessa atividade, e que esta forma de trabalho não está a margem do capitalismo.

Buscou-se entender o valor atribuído ao trabalho informal para todos aqueles que estão inseridos nele, seja direta ou indiretamente. E porquê optaram por escolher trabalhar informalmente e o que consideram do trabalho informal.

Abordou-se também o conteúdo relacionado ao trabalho informal e suas precariedades. Destacou-se também situações onde, apesar de muitas adversidades, existem pessoas que conseguem se destacar neste ambiente tão hostil e competitivo. Pesquisou-se sobre como os grandes pensadores, dentre eles Ricardo Antunes \cite{antunes1999sentidos}; Maria Augusta Tavares \cite{augusta}; G. Noronha \cite{noronha2003informal}; Robert Castel \cite{castel1998metamorfoses} encaram o tema. Também buscou-se informações relevantes colhidas através do censo do IBGE para completar o entedimento do trabalho informal e verificar quais são os fatores que o tornam precarizado e como ele ocorre na prática.

Portanto, será mostrada ao longo da pesquisa a ligação existente entre informalidade e precarização dentro do trabalho informal, assim como seus fatores positivos e negativos, perfil de pessoas que trabalham informalmente no Maracanaú e o que mudou na vida dessas pessoas que começaram a trabalhar desta maneira.

\chapter{Definições e Conceitos Sobre o Trabalho Informal E Precarização}

Aqui abordaremos o trabalho informal sob a perspectiva de grandes intelectuais dentre eles Ricardo Antunes e Robert Castel para tentar deixar mais claro algumas questões relacionadas ao trabalho informal e a precariedade contida no mesmo. A precariedade está inserida dentro do trabalho informal e em seu conceito estão inseridas muitas questões pertinentes que mostram a diversidade contida dentro dela, que trata não apenas de pessoas que trabalham na informalidade de uma forma honesta ou mesmo criminosa.

Primeiramente faz-se necessário entender que as relações existentes no trabalho informal não estão à margem do sistema capitalista, pelo contrário só existe porque o sistema permite que tal forma de organização seja produzida e reproduzida diariamente. Considerado como sinônimo de atraso e que seria eliminado com a evolução do capitalismo, o setor informal mostrou-se através de seus mecanismos que ele não seria facilmente excluído, mensurado ou controlado na sua totalidade.

O setor informal é de muita importância para o próprio capitalismo servindo
até mesmo de proteção social. Contrariando a teoria da subordinação \footnote{
Teoria formulada em 1980, segundo a qual o setor informal é uma forma de
produção subordinada e intersticial à produção capitalista. Nessa visão, o
espaço econômico onde o setor informal atua é destruído, criado e recriado
pelo movimento da acumulação capitalista. Paulo Renato C. Souza, Salário e
emprego em economias atrasadas.Campinas: Unicamp/ IE, 1999} sobre a crescente
expansão do trabalho informal, fica explicito que o setor informal não se
trata apenas de atividades que proporcionam um meio de sobrevivência.

\begin{citacao}
o capital necessita cada vez menos do trabalho estável e cada vez mais das mais 
diversificadas formas de trabalho parcial ou part-time, terceirizado, que são em 
escala crescente, parte constitutiva do processo de produção capitalista. \cite{antunes2011modos}
\end{citacao}

O trabalho como forma de inserção social é mostrado no relatório de Boisionat
\footnote{Em 1995, o Relatório Boissonat, concluiu que, no horizonte de vinte
anos o emprego continuaria sendo um meio essencial de inserção social.
Portanto, se a tecnologia economiza trabalho, é melhor desdobrar os empregos
existentes para que todos tenha um, do que dá-los a uns e privar
permanentemente outros.} que é necessário que todos tenham um trabalho para
não serem excluídos no sistema capitalista modernizado, além de ser muito
importante para a inserção social.

Através da flexibilização do trabalho \footnote{A flexibilização do trabalho
basicamente é uma nova condição para que a mão de obra consiga um emprego nos
dias atuais, porque é necessário que se desempenhe mais de uma função dentro
da empresa para não ser subsitituido por outra pessoa que aceite desempenhar
mais de uma função, é necessário ser flexível e aceitar desempenhar várias
funções dentro da empresa.} podemos notar que os defensores do capital assumem
a tarefa de mascarar as contradições do capitalismo e sempre procuram realçar
sua superfície para acharmos qu e é a sua essência. Por meio de suas formas de
exploração é possível enganar a mão de obra assalariada lhe dando uma falsa
autonomia em alguns empregos exercidos, que é marcada por um trabalho feito
por resultados.

Neste contexto de flexibilização das formas de trabalho fica cada vez mais
difícil acreditar nas estatísticas acerca do mercado de trabalho porque está
cada vez mais complicado identificar emprego e desemprego no contexto atual. O
desemprego na contemporaneidade se tornou algo complexo devido aos mecanismos
que o próprio capitalismo permite como, por exemplo, o trabalho informal, nas
suas mais variadas formas desde que não seja de estrita sobrevivência. Fica
cada vez mais difícil identificar quem realmente é desempregado e quem não é,
a verdade é que o que continua é a exploração do trabalho com varias
estratégias que mascaram a realidade social em que vivemos e seja qual for à
organização de trabalho no sistema capitalista não devemos esquecer que o
lucro, ou seja, a mais-valia continua sendo o foco principal.

\begin{citacao}
A lógica do sistema produtor de mercadorias vem convertendo a concorrência e a busca da produtividade num processo destrutivo que tem gerado uma imensa precarização do trabalho e aumento monumental do exército industrial de reserva, do número de desempregados. (ANTUNES, pág 18, 2009) 
\end{citacao}

\cite{antunes2009infoproletarios} nos alerta que a concorrência é um grande vilão para a
grande massa de trabalhadores que, vendem sua força de trabalho, porque essa
concorrência transforma a produtividade em um processo destrutivo e com isso
quem vai sofrer as piores consequências serão os operários que serão cada vez
mais explorados, e assim o capitalista busca vencer a concorrência que existe
entre empresas ou empregadores dentro do processo existente no sistema
capitalista, fazendo os donos do capital enriquecer cada vez mais.

Citamos o desemprego na contemporaneidade porque ele se tornou algo complexo pois existem outras atividades de trabalho realizadas por diversos brasileiros que não estão inseridos no mercado de trabalho formal, além de que o próprio capitalismo permite como, por exemplo, a existência do trabalho informal, nas suas mais variadas formas. Fica cada vez mais difícil identificar quem realmente é desempregado e quem não é, a verdade é que o quê continua é a exploração do trabalho, seja ele formal ou informal. O fato é que existem varias estratégias que mascaram a realidade no trabalho que estamos inseridos e seja qual for à organização de trabalho no sistema capitalista não devemos esquecer que o lucro, ou seja, a mais-valia continua sendo o foco principal.

Nossa missão é mostrar a precarização dentro do trabalho informal, iremos dar mais ênfase nessa questão socialmente relevante para o entendimento e conhecimento de todos.

Por conseqüência, quando um ou alguns trabalhadores procuram reivindicar melhores condições de trabalho, algumas vezes, eles acabam sendo demitidos, aumentando o índice de desemprego. Contribuindo ainda mais para o sistema capitalista, que necessita que existam muitos desempregados para poder explorar cada vez mais quem só tem sua força de trabalho para vender.

Então a partir dessa característica do capitalismo que permite que muitos fiquem desempregados, alguns buscam outro meio de sobrevivência para realizar seus objetivos individuais. Isso faz com que esses trabalhadores que estão desempregados e não conseguem retornar ao trabalho formal, ou seja, com a verdadeira proteção social e estatal regida principalmente pelas CLT (consolidação das leis trabalhistas), busquem uma atividade econômica Informal para viver. 

Para a grande maioria dos brasileiros, o trabalho informal está associado ao ``sem carteira assinada''. Não que essas pessoas estejam completamente erradas, mas o trabalho informal tem mais particularidades. Porém, até em pesquisas realizadas por entidades de pesquisa do governo brasileiro, o trabalho informal é sim definido como um trabalho sem carteira.

Muitos trabalhos com carteira assinada são extremamente precários, a grande maiooria dos postos de trabalhos existentes em nosso país não valoriza o trabalhador assalariado que dedica toda sua energia e empenho para garantir o lucro da empresa. Então muitos acabam migrando para o trabalho informal na expectativa de ser mais bem remunerado.

A grande massa busca no trabalho informal uma melhoria na condição de vida, principalmente porque muitas atividades dele são de baixos investimentos. Essas pessoas buscam condições mínimas de dignidade humana, rendimento e autonomia, mesmo que essa autonomia não seja total.

Segundo \cite{noronha2003informal} a palavra informalidade é bastante ampla e tem diversos conceitos subentendidos, então ele afirma que seria mais interessante deixar de falar ou escrever a palavra informalidade no tema do trabalho informal e passarmos a entender e falar em contratos atípicos, porque segundo sua visão seria mais contextual para o tema específico e de melhor entendimento de todos que fossem pesquisar, debater ou até mesmo ler algo relacionado com o trabalho informal, principalmente quando estamos falando de contratos de trabalho no Brasil.

É preciso ser bastante cauteloso quando falamos de trabalho informal no contexto brasileiro até porque cada país detém suas particularidades com relação ao tema abordado, então não devemos cometer o equivoco de achar que sabemos todos os conceitos e definições da palavra trabalho informal, existem muitas definições e regras especificas para cada contexto então, por exemplo, aqui no Brasil algo que é considerado informal pouco tempo depois pode ser considerado formal ou legal, isso porque o capitalismo está em constante mudança e sempre é preciso adaptar novas formas de trabalho que se encaixem no sistema capitalista para que se possa cobrar mais carga tributária.

Como o capitalismo está em constante mudança é bastante complicado compreender e destrinchar sua lógica estrutural mais especificadamente, não a lógica que é repassada por meio de uma aparência que busca mascarar seu sistema estrutural, no entanto é preciso entender que mesmo que sejam entendidos e compreendidos muitos conceitos inseridos no trabalho e na economia informal é necessário ter consciência de que não existe uma verdade única, até porque no contexto de humanidades, sabemos que as ciências sócias não são feitas de verdades absolutas e incontestáveis e consequentemente não podem prever o futuro que o trabalho informal ira percorrer ao longo dos próximos anos no Brasil, só podemos supor.

O sistema capitalista, com toda sua estrutura permite o que chamamos de trabalho informal ou até mesmo o que Noronha chama de contratos atípicos. Nos trabalhos informais ou contratos atípicos existe uma grande massa de trabalhadores e trabalhadoras que vivem desse trabalho considerado um trabalho precarizado. Existe sim uma forte relação entre trabalho informal e precarização do ponto de vista que os trabalhadores informais não possuem direitos trabalhistas fundamentais, como férias remuneradas, décimo terceiro salário, licença maternidade etc. 

Verdade é que o trabalho informal no Brasil é para a grande maioria aquele trabalho realizado sem carteira assinada e se o mesmo é realizado sem possuir formalidade acaba sendo precarizado porque quem exerce essa função de ser trabalhador informal é muito mais explorado nas suas atividades e não possui pelo menos um sindicato que possa intervir para garantir seus direitos mínimos. O trabalho informal é considerado precarizado devido as adversidades contidas dentro do mesmo, que o transforma numa atividade trabalhísticas complexa do ponto de vista das condições mínimas que um trabalhador possa desenvolver uma atividade que seja bem realizada e que possa garantir sua integridade física e moral. 

Esses trabalhadores que só possuem sua força de trabalho como meio de inclusão socioeconômica e como não possuem muita qualificação escolar acabam optando por algo considerado mais fácil ou até mesmo uma saída para ganhar dinheiro.

Sendo assim é preciso deixar explicito que através de nossas próprias políticas e através do próprio sistema capitalista foi criado um sistema precarizado de postos de trabalho, que desvaloriza cada vez mais o cidadão e busca somente explora-lo demasiadamente, só se importando como fazer a mão de obra ficar mais e mais barateada e inventar e reutilizar mecanismos passados (que deram certo) para intensificar a exploração e a busca por mais-valia.

Sabemos que a desigualdade entre as pessoas inseridas na lógica do mercado de trabalho brasileiro não existe apenas em nosso país, mas precisamos olhar para nós mesmos e buscarmos métodos para diminuir esse problema estrutural com ações eficazes para não deixar cada vez mais precarizado nossos trabalhadores que estão em ocupações que ficam na parte inferior da pirâmide do mercado econômico brasileiro, mas são de extrema importância para o desenvolvimento do sistema econômico do país.

Falar de trabalho é sempre complexo, porque respiramos essa palavra quase que em cem por cento de nossa vida, seja numa conversa entre amigos, seja quando escolhemos ingressar numa faculdade, curso profissionalizante, curso técnico, ou até mesmo quando estamos tentando conquistar alguém, por isso é preciso tomar cuidado quando falamos de trabalho precário até porque estamos falando de pessoas que mesmo sem entenderem que realizam essas atividades estão envolvidas nesse meio e é preciso respeitá-las sem fazer juízo de valor.
 
O sistema capitalista utiliza a classe de trabalhadores menos favorecidas para aumentar o lucro do mercado econômico brasileiro, e assim ser visto no exterior como um país que está em constante desenvolvimento e capaz de fazer parte das melhores e mais bem conceituada entidades externas e assim fazerem bons acordos com países centrais. No entanto a verdade é que o mais importante é somente explorar e conseguir obter o máximo de mais-valia possível de cada trabalhador.

Então como o sistema capitalista nos mostra que não existem oportunidades para todos e nossa grande maioria aceita esse dogma nem todos são considerados pessoas, (no sentido sociologicamente atribuído a palavra) e nem todos conseguirão realizar seus sonhos por mais simples que sejam. Por isso é necessário buscar e encontrar algumas formas de sobrevivência e não apenas de sobrevivência como também encontrar uma forma de se tornar pessoa com todos os atributos que enxergamos em alguém que é considerado pessoa na lógica do nosso sistema. Então o trabalho informal surge como forma de fazerem aqueles que foram deixados de lado pelo trabalho formal continuarem lutando por seus objetivos tanto pessoais como profissionais.

\begin{citacao}
Em sociedades democráticas a lei é, por definição, justa. Caso não seja, deve ser mudada, mas nunca desprezada. Contudo, muitos contratos considerados justos por determinados grupos não são previstos em lei ou são francamente ilegais. Além disso, no Brasil, popularmente, o trabalho ``informal'' típico pode ser entendido, se não como ``justo'', ao menos como ``aceitável'', e certamente não é considerado ``ilegal'' a menos que se trate de crime (em geral comércio de produtos ilegais) e não apenas um contrato ilícito. (NORONHA, pág. 121, 2003)
\end{citacao}

No Brasil o trabalho informal está diretamente ligado à atividade as margens da lei tanto por não existir leis especificam que abordem o trabalho informal, como também por existir atividades consideradas criminosas dentro do trabalho informal, no entanto é preciso enfatizar que o trabalho informal pode estar à margem da lei, mas não a margem do sistema capitalista, porque o próprio sistema permite a existência dessas atividades.

Então é bastante complexa essa relação entre forma e informal, legal e ilegal. O próprio sistema que te oprime e que não da uma proteção e uma oportunidade de vida melhor, permite uma meio que se torne uma saída que muda a vida de muitos, porém não de todos, nem de todos que estão inseridos no trabalho e na economia informal, como nem de todos que são excluídos pelo trabalho formal e tentaram o trabalho informal como beneficio pessoal e familiar.

Cada um que busca uma oportunidade nesse sistema vai encontrar diversas dificuldades, algumas parecidas com as mesmas do trabalho formal, algumas distintas. Mas o mundo do trabalho informal com certeza não é para todos é preciso de uma série de atributos que vão desde uma boa argumentação e persuasão até encontrar uma atividade informal rentável que lhe proporcione uma melhoria significativa de vida.
Então é necessário mostrar, que mesmo com as imensas dificuldades existentes por cada um que trabalha informalmente o trabalho informal é considerado de certa forma um meio de proteção social para estes trabalhadores que estão nesta atividade. Porque através dessa atividade informal de trabalho o indivíduo consegue retornar a ter um lugar na economia do país e conseguem ter as mínimas condições de dignidade para alguém que está inserido no sistema capitalista do Brasil.

O fato é que mesmo que seja considerado como marginalizado e precário o trabalho informal é tido como algo bom para sociedade brasileira que está inserida nessa atividade e que necessita de uma ocupação para dar continuidade em sua vida pessoal e profissional buscando e realizando seus objetivos mesmo que estejam esquecidos pela lei. Mas não tão bom para o país que deixa de arrecadar impostos dessas atividades, por não regulamentar essas atividades que tem um imenso impacto na economia geral do Brasil.

No Brasil o trabalho informal está relacionado às atividades que não possuem uma regulamentação governamental, mas mesmo assim é relatada sua existência em alguns órgãos de pesquisa como no IBGE. Mesmo que existam diversas atividades no mundo do trabalho in-formal brasileiro todas essas atividades são consideradas ilegais para as leis do Brasil (Por falta de legislação especifica), porém sabemos que hoje no contexto atual uma atividade que é considerada ilegal possa mudar de contexto e passaremos a ter uma legislação e uma regulamentação que beneficie muitas atividades que são informais, porém não são ilegais.
Existem sim muitas atividades informais que são ilegais no sentido de serem criminosas, mas também existem diversas atividades que são informais, mas não são criminosas e são essas atividades que devem ser regulamentadas para poder mudar esse contexto do trabalho informal brasileiro que é tido sempre como uma atividade marginalizada e precarizada.

Para melhorar a situação de todos que precisam do trabalho informal para sobreviver, ajudar a família, complementar a renda, se sentirem pessoas úteis para sociedade, possuírem capital econômico, realização dos objetivos e consequentemente melhorar o mercado brasileiro e sua economia. Porém para que esses benéficos sejam conquistados é preciso políticas públicas eficazes para que se resolva essa temática.
Com sua imensa complexidade o trabalho e a economia informal são estudados e considerados como algo ruim para a economia brasileira(para alguns intelectuais), porém não podemos generalizar e esquecer que existem pessoas que estão sendo beneficiadas as margens da lei, mas não as margens do opressor sistema capitalista. O sistema permite a existência da atividade informal, mas é prejudicado por não conseguir regulamentar esse e isso é tido como um problema estrutural que dificilmente vai mudar por completo devido a grande dimensão de atividades existentes dentro do trabalho informal.

Cada um que trabalha informalmente tem diversas dificuldades dentre as quais podemos citar: excesso de horas trabalhadas, dificuldade para controlar as finanças (no caso de quem é investidor), não tiram férias remunerada, não existem folga remunerada, não existe como já sabemos legislação que procure proteger o trabalhador, não podem ficar doente, muitas vezes falta de apoio na atividade exercida, nenhuma estabilidade etc.

Outro problema gerado por falta de uma regulação do mercado no trabalho informal no Brasil é o fato de que as estatísticas podem e devem estar erradas em muitas pesquisas, porque se em nosso país carteira assinada é sinônimo de estar empregado e sem carteira assinada é não estar empregado, fica subentendido que não existe outra forma de emprego se não o trabalho formal, porém na realidade não é bem assim que acontece. Então todos esses trabalhadores que se empregam em atividades informais são considerados desempregados mesmo que estejam trabalhando e consumindo diariamente. Por esse e outros fatores é sempre necessário levar em consideração o trabalho informal no Brasil para se entender melhor várias questões dentre elas o desemprego estrutural.
Para \cite{augusta} ``Não é o operário que usa os meios de produção é o meio de produção que usa o operário''. Podemos perceber como o capitalismo se comporta quando se trata de mão de obra de trabalho, percebe-se através dessa simples frase uma realidade bem clara quando estudamos o trabalho seja formal ou informal.

Em um país com oportunidades diferentes para seus habitantes a classe subalterna sempre fica esquecida quando é preciso que alguém seja beneficiado, por isso esses que são esquecidos por um país com estruturas que são difíceis de serem mudadas, tentam de alguma maneira ir atrás seus objetivos, mesmo com todas as dificuldades encontradas por quem está na classe inferior.
No contexto estudado a classe operaria sempre é quem fica com os piores empregos, com a pior educação, com os piores serviços essenciais para uma vida digna, porém mesmo com todas essas adversidades essa categoria de trabalhadores é uma categoria guerreira que luta diariamente por uma condição de ganhar uma melhor remuneração sem se preocupar com a precarização do trabalho exercido e enfrentando de frente esse problema que é a precariedade do trabalho informal.

Mas com tanta desvalorização tanto como seres humanos que precisam ter direitos básicos e também como vendedores de força de trabalho, o proletariado brasileiro não fica apenas esperando por políticas sociais eficazes, eles correm atrás de melhorias de vida através de uma atividade realizada com inúmeras adversidades, mas que se tornou o único meio para muitos de sobreviver e lutar por uma condição melhor.
O trabalho informal e sua economia não são importantes para as pessoas que não precisam dessa atividade; porém para quem está dentro dessa ocupação ele é tido como algo bom tanto sociologicamente falando como economicamente. Fazendo muitas pessoas encontrarem uma nova ocupação e também um novo meio de interação social que é fundamental na vida de qualquer ser humano, servindo de proteção para os menos favorecidos e uma forma de mostrar que o sistema capitalista é tão complexo que ele permite que o proletariado possa se auto-ajudar e ao mesmo tempo de alguma forma prejudicar o desenvolvimento econômico do país.

\begin{citacao}
O salário reconhece e remunera o trabalho ``em geral'', isto é, atividades potencialmente úteis para todos. Assim na sociedade contemporânea, e para a maioria de seus membros é o fundamento de sua cidadania econômica. Também está no princípio da cidadania social: esse trabalho representa a participação de cada um numa sociedade. É assim o ponto médio concreto sobre o qual se constroem direitos e deveres sociais, responsabilidades e reconhecimento, ao mesmo tempo que sujeições e coerções. (CASTEL, pág 581, 2010)
\end{citacao}

Além de tudo o que já foi destacado é importante salientar que em qualquer sociedade existe algo que é de extrema importância, no caso, o salário, porque dentro do salário está inserida diversas questões sociais que ajudam cada individuo a entender melhor sua vida, porém é necessário que essa importância seja notada por parte do trabalhador subalterno para que não troque sua força de trabalho por uma remuneração que não revela verdadeiramente o sofrimento e a dedicação contida no seu trabalho.

Através do direito do trabalho se conseguiu alguns benefícios para os trabalhadores, mesmo que com algumas fragilidades (no caso a falta de fiscalização por parte do Estado Social).O direito trabalhista melhorou muito a vida de muitos trabalhadores principalmente em relação as demissões, fez com que a relação entre empregado e empregador se equilibrasse porque se tornou necessário um motivo sério para demitir.


\chapter{O Município de Maracanaú}

Neste segundo capítulo será feita uma analise que nos mostrará um pouco sobre o município de Maracanaú,
saberemos as principais características desse município, clima, economia, história, quem é o prefeito etc.

\section{Aspectos gerais de Maracanaú}

A partir de uma analise de dados sobre Maracanaú disponível no Instituo Brasileiro de Geografia e Estatística \cite{demografico2000resultados}, 
conheceremos um pouco mais sobre este município que abriga vários trabalhadores informais. 

Maracanaú fazia parte do município de Maranguape até que em 1983 conseguiu torna-se independente. Através da Lei 
10.811 que determinou a criação deste grande município que é Maracanaú. E o significado do nome dado a este novo 
município que foi criado em 1983 quer dizer ``Lugar Onde bebem as Maracanãs'' que é uma palavra de origem Tupi.
Com relação a sua extensão Maracanaú possui uma área territorial de 105,70$km^2$ e possui um clima predominante 
tropical quente sub-úmido, com temperatura média entre 26$^{\circ}C$ e 28${\circ}C$ e um relevo de tabuleiros pré-litoraneos

Maracanaú tem sua divisão político-administrativa nos distritos de Maracanaú e Pajuçara. Sua regionalização o 
situa como município que faz parte da região metropolitana de fortaleza. No que desrespeita aos aspectos 
demográficos podemos informar que a população que reside em Maracanaú é de aproximadamente duzentos mil 
habitantes, sendo que em sua maioria é formada por crianças e jovens com idade de 0 até 24 anos e dentre 
toda essa população a maioria situa-se na zona urbana e a menor parte situa-se na zona rural.E tem como 
prefeito Firmo Camurça.

Na área da saúde, Maracanaú possui um total de 53 unidades de saúde ligada ao sistema único de saúde(SUS) 
sendo dividido entre sedes públicas  que são no número de 44 e sedes privadas sendo o número de 9 unidades.
Possui dentre suas unidades ligadas ao SUS 3 hospitais Geral, 1 posto de saúde,9 clinicas especializadas 
etc. Possui um programa que em que os agentes de saúde acompanham crianças de até 4 anos de idade. 

Seus principais indicadores de saúde são: 2,45 médicos a cada mil habitantes,0,32 dentistas a cada mil
habitantes, 1,25 leitos para cada mil habitantes, 0,25 unidade de saúde para cada mil habitantes e uma 
taxa de mortalidade de 9,0 para cada mil crianças nascidas vivas.

Na Educação o número total de professores que atuam dentro do município é de 2.766 distribuídos em escolas 
federais, estaduais e municipais e particulares. Dentre essas escolas citadas os alunos que usufruem o
direito de estudar e possuírem a sua disponibilidade alguns equipamentos essenciais dentre eles bibliotecas,
salas de aula em condições de ter aula e laboratórios de informática. Os indicadores mostram a população 
alfabetizada de Maracanaú é de aproximadamente 140 mil pessoas com mais de 15 anos.

Com relação ao emprego e renda o trabalho formal compreende uma população de aproximadamente 50 mil
trabalhadores que atuam dentro do município distribuídos dentre as principais atividades: Comércio, 
Indústria de transformação, construção civil, serviços e administração pública.

A economia do município corresponde ao PIB a preços de mercado (R\$ mil) de 3.121.055 PIB per capita(R\$ 1,00) 
15.620. Agropecuária 0,12, Indústria 57,93 e serviços 41,95. A finança pública tem uma receita total de 284.731 
e despesas de 304.382.

Com relação ao principal foco da pesquisa, os trabalhadores informais segundo uma pesquisa realizada em 2010 
realizada em Maracanaú mostram que 25,01 de cada 100 habitantes trabalha informalmente com idade entre dez anos 
ou mais e estão incluídos nesses dados aprendizes ou estagiários sem remuneração.

\chapter{O Cotidiano dos Trabalhadores Informais de Maracanaú}

 Neste capitulo abordaremos o assunto a partir de uma etnografia e algumas entrevistas. Escolhemos 
 como objeto principal para serem entrevistados os trabalhadores informais da Ceasa (Central de 
 abastecimento do Ceará), entrevististamos alguns trabalhadores e ex-trabalhadores informais da 
 Ceasa. Procuramos entrevistar aqueles que estavam trabalhando na total informalidade e com funções
 totalmente degradantes de trabalho.
 
 \section{O depoimento dos trabalhadores informais sobre o cotidiano de trabalho na Ceasa}
 
 Nosso primeiro entrevistado se chama Jefferson Duarte, masculino, pardo, 21, solteiro, ensino médio 
 incompleto, não tem filho e ainda mora com os pais, Jefferson nos disse que trabalhou na Ceasa 
 durante três anos, onde se vendiam uvas. 
 
 Ele trabalhava de Segunda-feira até Sábado e alguns Domingos.Existia uma divisão do trabalho conforme 
 os dias da semana, dia de segunda-feira e quinta-feira como é dia de chegar carregamento e de distribuir 
 a mercadoria para os clientes a carga horária de trabalho se tornava maior, começava as duas horas 
 da manhã até as 4 da tarde, totalizando quatorze horas de trabalho diário. 
 
 Terça-feira, Quarta-feira, Sexta-feira e Sábado o horário era de quatro
 da manhã até quatro da tarde. O serviço é basicamente o mesmo, porém com uma carga horária inferior
 aos dias anteriormente citados, então o que se fazia era organizar a mercadoria e pegar o caminhão
 e distribuir para os clientes. E depois que chegava as quatro da manhã só iam comer as nove horas
 da manhã. 
 
 Cada trabalhador ganhava quinze reais de almoço e cinco reais de merenda, e não existia 
 uma hora certa nem para almoçar nem para merendar cada um comia quando o trabalho aliviava um pouco 
 e sempre eram de dois em dois para comer (isso em uma equipe de seis trabalhadores), os outros 
 continuavam trabalhando. Sendo que de remuneração cada um recebia duzentos reais por semana. E 
 sendo que não existia muito critério para contratação de novos funcionários se algum amigo te 
 levasse e estivesse precisando de alguém em tal lugar era só se apresentar e o dono já te mandava 
 trabalhar, sem nem mesmo saber idade, qualificação etc.

 No aspecto relacionado as dificuldades do trabalho Jefferson nos relatou que a desempenhava várias
 funções ao mesmo tempo, já sofreu acidente no trabalho, não tinha folga remunerada, não tinha
 férias, e se faltar mesmo com atestado a remuneração semanal era descontada, porque o patrão 
 achava que os atestados eram falsos. 
 
 A relação com o patrão era uma relação complicada se errasse
 em algum serviço o trabalhador levava logo o nome de burro e outras palavras de baixo calão. 
 As maiores dificuldades enfrentadas segundo este depoimento de Jefferson ``é porque todo trabalho 
 é braçal, além de descarregar as mercadorias no obro ainda tinha que levar para outro lugar em
 um carrinho carregando sozinho.'' além de tudo ser manual cada trabalhador realizava várias atividades
 dentre as quais: descarregar o caminhão, pesar as frutas, selecionar as frutas boas e ruins e depois 
 colocar todas as caixas secas de volta no caminhão.

 Ao longo desses três anos Jefferson passou por inúmeras dificuldades relacionadas a sua situação no
 trabalho ele nos relatou um acidente de trabalho: ``Eu estava levando um carregamento que tinha quinze 
 caixas no carrinho mas só cabiam dez caixas, mesmo assim o patrão me obrigou a levar.. no caminho eu 
 enganchei minha mão no carregamento e acabei tendo um corte feio. Depois que isso aconteceu o meu 
 patrão só fez jogar um pouco de água gelada na minha mão e colocar uma atadura e depois disse que já
 podia voltar para o trabalho. Nem fui para o hospital e nem fui indenizado''. Outra dificuldade relatada
 foi que no final de ano como existem um movimento maior, Jefferson acabou saindo de casa para trabalhar
 as duas horas da madrugada e chegando em casa apenas as dez horas da noite.

 Outro problema que Jefferson nos relata do trabalho era a grande quantidade de insetos e animais que 
 contaminavam o ambiente o que se tornava mais fácil de ficar doente. Tinha muita lama e passava vários
 animais que com certeza contaminava tudo. Jefferson nos diz: ``E também mais um problema era que a 
 grande quantidade de trabalhadores com idade inferior a dezoito anos e que a fiscalização era uma vez
 perdida e quando chegavam alguns dos trabalhadores de menor se escondiam, e os outros a fiscalização
 só perguntava idade sem pedir documento nenhum''. 
 
 Outro fato importante é que dentre os trabalhadores 
 existia um vocabulário comum quando um o ou alguns trabalhadores desviavam a carga para si, o que para 
 o a maioria do senso comum é chamado de roubo ou furto, para eles é chamado galinha morta, gol, golaço.
 
 Jefferson tinha uma real noção de que seu trabalho era um trabalho extremamente precário, no entanto
 a facilidade com que se conseguia uma vaga de trabalho fazia com que ele acabasse sempre indo para
 Ceasa quando precisava de dinheiro. Em suas palavras Jefferson diz: “A situação era precária até demais,
 se sentia explorado sim, mas não queria nem saber, o importante era estar com dinheiro no bolso sábado
 (dia do pagamento)”.

 Apesar de Duarte admitir que o trabalho era explorador e poucos aguentavam ficar ele continuava 
 trabalhando, porém um dia ele sentiu um problema de saúde e foi até ao médico. Ele relatou esse fato
 assim: ``Um dia passei mal e fui para o médico e ele me perguntou qual era meu trabalho. eu disse que 
 trabalhava na Ceasa expliquei direitinho como era então ele me disse que eu estava trabalhando muito 
 e descansando pouco e se continuasse assim só teria mais seis meses de vida, então eu optei por sair 
 da Ceasa e nesse momento foi que eu cai na real e vi que não era aquilo que eu queria para o resto da 
 minha vida então voltei a estudar''. Portanto, percebe-se que mesmo quando alguém não se importa em ser
 explorado o corpo não consegue aguentar.

 Entrevistei outro ex-trabalhador informal da Ceasa: Cristiano Martins, solteiro, vinte dois anos, 
 pardo, ensino médio completo e não tem nenhum filho. Ele trabalhou na Ceasa durante um ano, quando 
 entrou tinha apenas dezenove anos e resolveu trabalhar lá porque precisava de dinheiro e não tinha 
 oportunidade no mercado formal e também porque não tinha experiência comprovada na carteira de trabalho
 e possuir pouca qualificação e capacitação que pudesse lhe proporcionar uma boa ocupação no mercado de
 trabalho e como a demanda por mão de obra e a facilidade para conseguir um trabalho informal na Ceasa
 é grande alguns amigos o convidaram a ir trabalhar lá e ele acabou aceitando.

 
 Cristiano descreve o trabalho que ele realizava na Ceasa como um trabalho extremamente difícil por 
 alguns motivos como acordar cedo, carregar muito peso e ser extremamente explorado, além de receber 
 pouco. Como Cristiano só trabalhava nos dias de terça-feira e quinta-feira só ganhava uma diária 
 correspondente no valor de trinta e cinco reais (isso porque era no setor que vendia predominantemente 
 morango, o preço da diária varia em cada local de trabalho), o horário de trabalho era de duas da manhã 
 até ao meio-dia, no entanto eventualmente ele entrava as nove da noite e só saia ao meio-dia (mesmo 
 assim continuava ganhando o mesmo valor da remuneração diária dos outros dias). A principal dificuldade
 que ele encontrou foi de acostumar com o horário de trabalho.

 No seu cotidiano de trabalho Cristiano Martins desenvolvia diversas atividades, desde limpar o caminhão
 até organizar e vender toda a mercadoria comercializada, a atividade trabalhista se tornava pesada porque
 a atividade desenvolvida era descarregar as caixas com morango e levar para outros compartimentos onde a
 mercadoria era armazenada ou para ser vendida, no entanto mesmo tendo como função principal descarregar 
 a mercadoria Cristiano nos diz que desenvolvia qualquer atividade que fosse pedida. 
 
 Com toda essa exploração
 e com tantas dificuldades um ponto positivo foi que Cristiano nunca sofreu nenhum acidente no trabalho, 
 o que é algo frequente entre os trabalhadores informais da Ceasa que trabalham descarregando os caminhões
 com frutas e outros alimentos. A seguir Cristiano nos relata que passou por uma situação complicada no 
 trabalho: ``Teve um dia que eu estava organizando as mercadorias e não tinha nenhum carrinho para levar
 para os outros compartimentos, então o encarregado me pediu para levar uma caixa na cabeça que pesava 
 aproximadamente uns cinquenta quilos, mas eu disse que não ia levar, até porque não tinha condição por 
 causa do meu porte físico, eu achei um total absurdo''.

 Cristiano se sentia totalmente explorado no trabalho que realizava, principalmente porque ele trabalhava
 muito e ganhava pouco, não era respeitado, não era valorizado de nenhuma maneira. Mas como ele precisava 
 muito do dinheiro ia somente por isso e com essa diária que ele ganhava custeava suas necessidades pessoais.

 Na Ceasa trabalham diversas pessoas, algumas com intenções de ganhar a dinheiro trabalhando honestamente
 e outros nem tanto, sabendo que é comum ocorrer alguns desvios e furto de mercadoria na Ceasa o entrevistado 
 nos diz que aonde ele trabalhou acontecia sim esse tipo de coisa, mas que não são todos os trabalhadores que
 costumam fazer isso e também depende de cada lugar, cada local tem sua particularidade para esses desvios,
 se o dono, por exemplo, não controla sua mercadoria fica mais fácil desviar, mas se ele controla existe 
 outras maneiras de burlar o controle do patrão. 
 
 Segundo o depoimento de Cristiano isso ocorria no seu setor
 de trabalho: ``Quando eu e o funcionário mais antigo fazíamos entrega no caminhão para outros locais fora da
 Ceasa o funcionário mais antigo, que era o responsável por toda mercadoria, acabava desviando algumas caixas,
 porque o dono do morango não tinha um controle rigoroso da mercadoria e isso facilitava o desvio''. Depois 
 desse depoimento entendemos que cada setor tem suas particularidades com relação ao desvio de carga, porém 
 não existe somente esse tipo de pessoa trabalhando na Ceasa, existem também trabalhadores honestos que 
 procuram melhorar de vida através dessa ocupação de trabalho.

 Os dois próximos entrevistados da pesquisa ainda continuam trabalhando na Ceasa. O primeiro deles é Alexandre
 Rabelo, pardo, vinte e um anos, solteiro, ensino médio incompleto. Alexandre começou a trabalhar na Ceasa com 
 dezenove anos e o que mais o motivou para ir trabalhar lá foi à falta de outras oportunidades de emprego e 
 também a necessidade por um trabalho rápido. 
 
 Os seus amigos lhe motivaram a ir trabalhar na Ceasa e ele acabou 
 aceitando. Alexandre trabalha de segunda-feira até sábado, com dois horários diferentes nos dias de terça-feira
 e quinta-feira ele entra às onze e meia da noite e sai as nove e meia da manhã. E nos outros dias que ele trabalha 
 o horário é de duas e meia da madrugada até as dez da manhã, seu ganho semanal é de duzentos e setenta e cinco reais. 
 Sua única folga é no domingo e teve suas primeiras férias depois de um ano de trabalho

O trabalho desenvolvido por Alexandre na Ceasa é de separar as frutas boas das 
frutas ruins e ele considera boa sua atividade de trabalho só acha ruim o fato 
de não ter tempo específico para merendar, é só merendar rapidamente e já voltar 
para o serviço e também quando falta algum outro funcionário, isso lhe faz 
trabalhar mais para compensar a ausência do seu colega de trabalho e é nesse 
momento que ele se sente explorado no trabalho. 

A única dificuldade encontrada 
por Alexandre foi se acostumar com o horário de trabalho e o trajeto 
casa-trabalho que lhe deixa inseguro devido ao fato de terem poucas pessoas no 
horário que ele sai para trabalhar, fora isso ele nos relata que o resto foi 
fácil.

 Com relação à experiência adquirida no trabalho ele nos diz que foi de extrema 
importância para sua vida porque o tornou mais responsável e lhe ajudou a 
conquistar muitos dos seus objetivos pessoais, além de poder ajudar sua mãe 
financeiramente, que era uma motivação particular de Alexandre.

 O segundo entrevistado é David Martins, pardo, dezoito anos, masculino, 
solteiro, ensino médio incompleto. Trabalha na Ceasa desde os dezesseis anos de 
idade. David optou por ir trabalhar na Ceasa porque estava precisando ajudar a 
família e precisava de um trabalho rápido, então seus amigos indicaram a Ceasa 
como um local que com facilidade se encontrava um emprego porque lá sempre 
precisa de mão de obra. 

Com a influência dos amigos David conseguiu o trabalho e 
começou rapidamente, já na segunda semana de trabalho estava adaptado ao 
serviço. Os dias de trabalho de David são as terças-feiras e quintas-feiras, com 
horário de entrada a uma hora da madrugada e horário de saída as dez horas da 
manhã, com uma remuneração diária de cinquenta reais, e cinco reais para custear 
sua merenda.

 Segundo o relato de David seu trabalho é carregar e descarregar os caminhões 
com as caixas de frutas e fazer as entregas, dentre outras atividades que sejam 
necessárias para o melhor desenvolvimento do serviço. Sua principal dificuldade 
no começo foi acostumar-se com o horário de trabalho, porque nunca tinha 
trabalhado em nenhum horário na verdade, mas com o passar dos anos isso se 
tornou algo normal e hoje não existe mais essa dificuldade. 

Além do horário 
David também passou por um momento complicado no começo do trabalho quando 
sofreu um acidente quando estava descarregando o caminhão com outro colega de 
trabalho e por um descuido seu colega acabou jogando uma caixa na sua cabeça que 
pesa aproximadamente uns quarenta quilos, depois desse ocorrido seu patrão só 
perguntou se ele queria ir para casa e não ofereceu nenhuma ajuda para leva-lo 
ao médico ou sequer o indenizou por esse acidente sofrido no serviço. Passando 
essas dificuldades no começo hoje David se considera totalmente adaptado ao 
serviço. 

David nos diz também que sua relação com os colegas de trabalho é boa 
e também com seu patrão o ambiente é sempre de descontração. Quando pergunto se 
ele se sente explorado no seu trabalho ele responde que sim e diz ``Não quero 
mais isso para mim e nem para ninguém, porque além do trabalho ser muito ruim o 
pior de tudo foi que atrapalhou meus estudos e eu parei no primeiro grau do 
ensino médio''. 

 Com relação às boas coisas que o trabalho lhe trouxe ele só consegue citar o 
dinheiro, que serve muito para ajudar sua família e custear seus interesses 
pessoais como comprar roupa e utilizar o dinheiro para sair com os amigos.

 O que conseguimos notar em relação a esses trabalhadores da Ceasa que foram 
entrevistados é que existe uma necessidade de ajudar a família e de se auto 
ajudar, então eles acabam aceitando o primeiro trabalho que aparece, sem se 
preocupar com as consequências futuras como, por exemplo, deixar os estudos para 
segundo plano. Mas a própria pressão familiar por uma ajuda financeira os faz 
optar por esse caminho. 

O fato é que a atividade de trabalho realizada é 
verdadeiramente o que menos importa para esses trabalhadores eles só se 
preocupam com a remuneração, mesmo que tenham que sofrer qualquer tipo de 
adversidade por parte da atividade exercida.

 Entrevistei outra pessoa que trabalha informalmente conhecida popularmente como 
Dona Dina,ela possui uma barraca de churrasco,que fica localizada no município 
de Maracanaú mais precisamente no bairro da Pajuçara . Dina como é popularmente 
conhecida tem cinquenta e sete anos de idade, divorciada, três filhos, é de cor 
parda, e só concluiu até a sétima série do ensino fundamental (fundamental 
incompleto). 

Ela nos relata um pouco da história de como tudo começou para hoje 
ela ser uma trabalhadora informal que consegue ganhar a vida e de certa maneira 
ter muitos benefícios com sua profissão. Tudo começou quando por acaso, o 
espetinho e o ponto já tinham um dono e esse dono era seu amigo, certo dia ele 
adoeceu e devido a sua doença ser um pouco grave, segundo dona Dina, ele 
ofereceu a barraca e o ponto para por cinco mil reais e disse que ela poderia 
pagar de qualquer forma, então ela aceitou a proposta e começou a tocar o 
investimento passou um ano trabalhando praticamente só e depois de um ano de 
muita dificuldade contratou um funcionário para ajuda-la (ela paga cem reais por 
semana para ele),passado essas dificuldades hoje ela vende em média cento e 
cinquenta espetinhos por dia, fora os refrigerantes, cerveja, cachaça e baião 
nos dia de sexta-feira. 

Ela não se queixa do trabalho a única adversidade para 
ela é que uma parte do trabalho (comercialização das mercadorias) é realizada no 
meio da rua, mas ressalta a importância do investimento para sua vida e descreve 
com algo essencial para sua existência pessoal e profissional. Há pouco tempo 
dona Dina contratou outro funcionário com a mesma remuneração do seu primeiro 
funcionário e ela relata que sua relação com os funcionários é boa, a única 
cobrança é com relação a higiene dos funcionários, ela dá folga apenas no 
domingo e no final do ano dá vinte dias de férias no mês de dezembro. E já faz 
três anos que ela está desenvolvendo está atividade e se considera muito 
satisfeita e realizada por ter essa profissão.

\section{Etnografia realizada sobre o cotidiano do trabalho informal}

No dia vinte e seis de junho de dois mil e quatorze realizei minha pesquisa 
etnográfica com dona Dina e seus dois funcionários. Cheguei à sua residência as 
sete e meia, horário em que ela e seus ajudantes começaram a cortar as carnes, 
fazer os temperos e preparar os espetos para serem comercializados mais tarde. 

Tudo é realizado em um compartimento especifico dentro de sua própria casa, que 
é onde todas as preparações dos alimentos ocorrem e também onde as carnes são 
separadas e onde estão os vários utensílios que são usados no serviço, mas antes 
de eu te chegado dona dina e seu marido já tinham ido até a Ceasa as quatro 
horas da manha comprar as carnes que serão preparadas para a comercialização. 

Todo procedimento é feito por Dina e seus ajudantes, ela primeiramente lava toda 
a carne e depois começa a cortar (serviço demorado e um pouco arriscado devido 
ao fato de que qualquer erro pode significar a perca de um dedo ou vários dedos 
da mão) toda carne é separa em varias vasilhas de plástico numa mesa separada 
apenas para realização desse serviço, separando cada carne no seu devido lugar e 
seus ajudantes vão colocando as carnes nos palitos de espeto. 

Com muita técnica 
eles colocam a carne no palito, mas mesmo com muito cuidado às vezes acabam 
furando os dedos em algum momento de descuido. Esse procedimento de corte e 
ajuste da carne acontece entre as sete e meia da manhã até as onze e meia da 
manhã. Depois desse procedimento realizado de manhã dona dina e seus ajudantes 
vão almoçar. Os ajudantes almoçam em suas respectivas casas e depois retornam ao 
serviço.

Os ajudantes voltam ao serviço às uma e meia da tarde e começam a organizar os 
objetos que serão levados até o ponto de comercio, que localiza-se 
aproximadamente uns trinta a quarenta metros da casa de Dina, em um terreno bem 
próximo da sua casa, começa a organização das mesas e cadeiras, copos, vasilhas 
com os espetos crus, copos descartáveis, sacolas, balde de água que serve para 
lavar as mãos tanto dos ajudantes quanto dos clientes, a churrasqueira, o carvão 
e vários pedaços de madeira e papelão, uma lona que serve para proteção contra o 
sol e as demais matérias que serão comercializados entre eles cachaça e 
refrigerante. 

Aproximadamente as duas horas da tarde os ajudantes saem para começar o trabalho 
no ponto de comércio e dona dina fica em casa descansando ou fazendo outras 
atividades que julgue ser necessária. Quando chegam, os ajudantes primeiramente 
montam a barraca e colocam as mesas e cadeiras por perto, mas ainda não as deixa 
organizada, depois os ajudantes montam a churrasqueira, colocam um cesto de lixo 
no chão e o balde de água em uma mesa separada. 

Ao término desse procedimento de 
ajustes os ajudantes colocam o carvão na churrasqueira e misturam com os pedaços 
de madeira e papelão para ascender o fogo. Eles têm um pouco de dificuldade para 
ascender o fogo por causa dos ventos, mas conseguem ao fazer esse procedimento 
eles me relatam que o ruim é a fumaça que faz e que a temperatura se eleva ainda 
mais. Depois de ascender o fogo eles começam a assar as carnes de quatorze em 
quatorze (que é o que cabe na churrasqueira) vão assando até terminarem tudo, 
esse procedimento dura em média duas horas ou duas horas e meia, enquanto isso a 
clientela é fraca, aparece poucas pessoas, uns para beber uma dose de cachaça e 
outros para comer um espetinho. 

Quando é aproximadamente umas cinco horas dona Dina chega e pede para os 
ajudantes organizarem as mesas e cadeiras, pouco tempo depois a clientela começa 
a surgir em maior número e em um descuido dona dina é enganada por um cliente 
que pede o espeto, come, porém sai de sem pagar, essa eventualidade acontece 
segundo ela e conseqüentemente isso lhe deixa bastante irritada, mas o serviço 
continua e a cliente chega de vários locais e de diversos meios de condução, a 
pé, de carro, de motos, de bicicleta acompanhada por outras pessoas ou não. 

Quando o movimento aumenta quem atendente quase toda a clientela é os ajudantes, 
dona Dina passa a maior parte do tempo conversando com alguns clientes ou 
amigos, uma vez ou outra ela atende algum cliente depois volta a ficar sentada 
conversando. Quando é aproximadamente umas seis e meia da noite o movimento 
chega ao auge, todas as mesas estão lotadas e várias pessoas pedem os diversos 
produtos oferecidos, isso permanece até umas sete e meia da noite, depois desse 
horário aparecem outros clientes, mas o movimento diminui bastante, mas o 
trabalho continua até às oito e meia da noite depois desse horário todo os 
materiais são recolhidos pelos ajudantes de dona Dina, primeiro eles começam a 
levar a cadeira, arrumam as cadeiras e levam nos braços mesmo, depois levam 
também as mesas, algumas são levadas em um carrinho de mão e outras são levadas 
nos braços de um dos ajudantes, quando os dois ajudantes terminam essa tarefa 
eles começam a organizar os materiais e guardam tudo em um compartimento dentro 
do carrinho de espeto, derrama-se a água suja que foi utilizada e também a água 
que restou e guardam o balde. 

Quando tudo está organizado os dois ajudantes levam os carrinhos de volta para 
casa de dona Dina, que fica bem próximo, e colocam dentro da sua garagem (lugar 
em que ficam os dois carrinhos do espeto, carvão e a churrasqueira), depois que 
tudo está na casa os espetinhos que sobram são colocados no freezer e segundo 
Dina ainda podem ser aproveitados até oito dias depois. É descarregada toda a 
mercadoria contida dentro do carrinho e colocada no compartimento da casa que 
fica os materiais específicos para o trabalho, para mais tarde serem lavados por 
dina e seu marido, depois de descarregarem as mercadorias os ajudantes de dona 
dina estão liberado do serviço e volta para suas casas, isso aproximadamente 
umas nove horas da noite para recomeçar novamente a jornada de trabalho às sete 
e meia da manhã do outro dia, depois que os funcionários se vão, Dina vai contar 
o dinheiro que foi ganho no dia trabalhado, que nesse dia deu cento e noventa e 
sete reais porque foram vendidos quarenta espetinhos que possuem um preço de 
dois reais e cinqüenta centavos e quinze espetinhos que valem três reais, dez 
refrigerantes de latinha foram vendidos a dois reais, doze doses de cachaça 
vendidas a um real e outras dez latinhas de cerveja vendidas a dois reais. 
Depois da contagem do dinheiro apurado o meu próprio expediente de trabalho 
etnográfico já está encerrado, isso aproximadamente umas nove e meia da noite.

Através de toda nossa pesquisa empírica podemos resaltar na pratica algumas 
questões que Robert Castel nos relata em sua obra intitulada de: A questão da 
metamorfose social.

\begin{citacao}
Assim o desemprego é seguramente,hoje, o risco social mais grave, o que tem os 
efeitos desestabilizadores e dissocializantes mais desastrosos para os que o 
sofrem. \cite{castel1998metamorfoses}
\end{citacao}

O que Castel relata em sua obra é comprovado em nossa pesquisa, porque existe 
esse medo do desemprego na prática e também essa realidade do desemprego que 
frequentemente nos aborrece. 

O fato é que o desemprego acaba fazendo muitas 
pessoas, em especial as de menor renda familiar, se submeterem em trabalhos 
totalmente precários, tanto do ponto de vista do salário quanto da atividade 
realizada e mesmo assim não se importam com essa condição em que se encontram.

Os efeitos são realmente desestabilizadores, e é função do sistema capitalista 
para fazer a massa trabalhadora ser cada vez menos valorizada e procurar 
submeter-se a qualquer trabalho para ajudar a família ou até mesmo sustentar a 
própria família, seja qual for o motivo a função do sistema capitalista é usar 
seus mecanismos para transformar o ser humano em Mao de obra desvalorizada para 
aumentar seu objetivo final. 

Não podemos mensurar os efeitos, mas eles existem e 
algumas provas deles estão nos relatos dos nossos entrevistados. Asseguir Castel 
nos diz:

\begin{citacao}
Mas o desemprego é a manifestação mais visível de uma transformação profunda da 
conjuntura do emprego. A precarização do trabalho constitui-lhe uma outra 
característica, menos espetacular põem ainda mais importante, sem dúvida. \cite{castel1998metamorfoses}
\end{citacao}

Quando pensamos que o desemprego é a pior de todas as situações que podemos 
enfrentar em nossas vidas, percebemos que a situação da precariedade do trabalho 
é também de extrema importância porque independentes se estão falando de 
trabalho formal ou informal a precariedade existe em cada um deles. E como foi 
mostrado em nossa etnografia e em nossas entrevistas essa precarização está 
presente no cotidiano do trabalhador informal de Maracanaú. Segundo Castel:

\begin{citacao}
Mas a empresa falha igualmente em sua função integradora em relação aos jovens. 
Elevando o nível das qualificações exigidas para a admissão ela desmonetariza 
uma força de trabalho antes mesmo que tenha começado a servir. Assim, jovens que 
há vinte anos teriam sido intergrados sem problemas a produção acham-se 
condenados a vagar de estágio em estágio ou de um pequeno serviço a outro. 
Porque a exigência de qualificação não corresponde sempre a imperativos 
técnicos. \cite{castel1998metamorfoses}
\end{citacao}

Castel nos relata na citação acima uma das problemáticas que percebemos ser ter 
maior importância para muitos trabalhadores informais de Maracanaú. Os jovens 
que trabalham na Ceasa informalmente geralmente procura oportunidade lá por não 
terem qualificação nenhuma para poder ingressar no mercado de trabalho formal, 
então como não servem para empresas porque não estão qualificados ou porque não 
possuem experiência esses jovens acabam encontrando no trabalho informal 
exercido na Ceasa e em outros lugares de Maracanaú a oportunidade que não 
encontraram no trabalho formal.

\bibliography{bib}

%\appendix
%\chapter{Isto é um apêndice}

Isto é um apêndice.  Lorem ipsum dolor sit amet, consectetur adipisicing elit,
sed do eiusmod tempor incididunt ut labore et dolore magna aliqua. Ut enim ad
minim veniam, quis nostrud exercitation ullamco laboris nisi ut aliquip ex ea
commodo consequat.  Duis aute irure dolor in reprehenderit in voluptate velit
esse cillum dolore eu fugiat nulla pariatur. Excepteur sint occaecat cupidatat
non proident, sunt in culpa qui officia deserunt mollit anim id est laborum.

\section{Subseção de um apêncice}

Subseção de um apêncice. Lorem ipsum dolor sit amet, consectetur adipisicing
elit, sed do eiusmod tempor incididunt ut labore et dolore magna aliqua. Ut
enim ad minim veniam, quis nostrud exercitation ullamco laboris nisi ut aliquip
ex ea commodo consequat.  Duis aute irure dolor in reprehenderit in voluptate
velit esse cillum dolore eu fugiat nulla pariatur. Excepteur sint occaecat
cupidatat non proident, sunt in culpa qui officia deserunt mollit anim id est
laborum.


\section{Outra subseção de um apêndice}

Outra subseção de um apêndice.  Lorem ipsum dolor sit amet, consectetur
adipisicing elit, sed do eiusmod tempor incididunt ut labore et dolore magna
aliqua. Ut enim ad minim veniam, quis nostrud exercitation ullamco laboris nisi
ut aliquip ex ea commodo consequat.  Duis aute irure dolor in reprehenderit in
voluptate velit esse cillum dolore eu fugiat nulla pariatur. Excepteur sint
occaecat cupidatat non proident, sunt in culpa qui officia deserunt mollit anim
id est laborum.




\end{document}

