\chapter{Considerações Finais}

Portanto, nossa pesquisa nos mostra primeiramente a grande quantidade de 
significados contido dentro do conceito da palavra informalidade, mais 
precisamente dentro da palavra trabalho informal, existem muitas atividades que 
não conseguimos mensurar, porém numa regra geral entendemos como o trabalho sem 
carteira assinada, essa é definição do senso comum brasileiro quando tratamos do 
assunto trabalho informal. Esse senso comum não está totalmente errado, porém 
existem outros conceitos que se encaixam dentro da informalidade. O conceito 
trabalho informal ainda não é um conceito em que a maioria dos intelectuais 
concordem mas para a maioria dos brasileiros significa trabalhar sem carteira 
assinada.

Em nossas pesquisas conseguimos concluir que o trabalho informal para muitos 
habitantes de Maracanaú significa uma saída para melhorar sua condição humana e 
social, significa fazer parte de um determinado grupo social, isto é o grupo que 
participa ativamente da economia do município, que através de sua força de 
trabalho sustenta a família ou ajuda a família. E isso é de extrema importância 
para o bom convívio social e isso torna ainda mais importante o trabalho, mesmo 
que esse trabalho seja informal e extremamente precário. Mas não esquecendo o 
que Maria Augusta (2002) nos diz “Deve-se, portanto, recusar a ideia de que o 
trabalho informal se restringe às atividades de sobrevivência”. 

Para alguns dos habitantes de Maracanaú o trabalho informal significa mais do 
que apenas um trabalho qualquer, que te ajuda a ter uma condição mínima para 
sobreviver e conseguir suprir suas necessidades básicas com o mínimo de 
qualidade. Significa aquele trabalho extremante bom, o trabalho que vos realiza 
como pessoa, é aquilo que vos satisfaz e que não encontraríamos em nenhum outro 
lugar. Tanto do ponto de vista de uma boa condição de trabalho e de uma boa 
remuneração quanto do ponto de vista de uma realização pessoal em que através 
desse trabalho o conforto chegou até seu próprio lar de uma forma até mesmo 
inesperada. O trabalho informal lhes proporcionou algo que sua pouca formação 
acadêmica não iria lhe proporcionar em uma vida toda de trabalho formal.

Percebemos que para alguns esse trabalho informal tido por precário por diversos 
motivos já citados aqui, como o excesso de horas trabalhadas, condições ruins de 
trabalho, falta de folga remunerada etc. É considerado algo muito melhor do que 
diversos trabalhos formais, claro que estamos falando das pessoas que se 
tornaram proprietária de investimentos no setor informal, porque te proporciona 
uma boa remuneração, não paga tributos etc. Porém até chegar a esse patamar de 
tornar algo que para a maioria serve como saída para encontrar um trabalho que 
se torne em algo extremamente lucrativo e bom é necessário persistência, 
determinação, dentre outros adjetivos que percebemos serem precisos ao longo, 
principalmente do nosso trabalho de pesquisa. 

Quando adentramos ao mundo do trabalho informal podemos notar não apenas a 
importância do ponto de vista da remuneração quanto dá integração social que 
Maria Augusta nos diz:

\begin{citacao}
A reconfiguração da empresa e do emprego é portadora de questões teóricas 
fundamentais para a sociedade, pois sendo o trabalho uma relação social, não há 
como separar a sua função econômica do contexto social em que o mesmo se 
desenvolve Tavares, 2002, pág 54.
\end{citacao}

Sem poder separar o trabalho do convívio social é cada vez mais clara a 
importância do trabalho na vida de qualquer individuo. Podemos, portanto notar 
que mesmo que seja informal nada diferencia com relação a essa integração social 
que o trabalho nos proporciona.
O trabalho informal em Maracanaú é e continua sendo algo que ajuda e facilita a 
vida de muitos habitantes e que como sabemos nem sempre se restringe a 
atividades de sobrevivência, alguns conseguem transformar em algo totalmente 
lucrativo e viver em boas condições de vida. Então percebe-se que nossa pesquisa 
mostrou na pratica o pensamento dos grandes intelectuais utilizados no trabalho 
em si, mostramos a importância do trabalho para essas pessoas que residem em 
Maracanaú, mostramos a definição conceitual de trabalho sem carteira assinada é 
relacionado com trabalho informal não só em Maracanaú mas também em todo o 
Brasil e mostramos na pratica o que Maria Augusta (2002)  nos diz a respeito do 
trabalho informal, que não é uma atividade que apenas o individuo faz para 
sobreviver, existem sim muitas pessoas em Maracanaú lucrando e usufruindo muito 
desse trabalho informal que para a grande maioria é sim muito precário, mas 
existe sim uma parcela de pessoas que conseguiram sobressair e se beneficiar 
muito através dele. 
Portanto, mesmo que seja precário informal e até mesmo ilegal. O trabalho 
informal é a fonte de renda e um dos poucos meio para que a população menos 
favorecida da sociedade. Mesmo que seja extremamente explorador no trabalho 
informal muitos moradores de Maracanaú conseguem melhorar de vida e realizar 
seus objetivos profissionais e pessoais, então concordamos com o Tavares e 
atestamos na prática que o esse trabalho não é apenas para sobrevivência. 
