\chapter{Conclusão}

Através de toda nossa pesquisa empírica podemos ressaltar na pratica algumas questões que Robert Castel \cite{castel1998metamorfoses} nos relata em sua obra intitulada de: A questão da metamorfose social.

\begin{citacao}
Assim o desemprego é seguramente,hoje, o risco social mais grave, o que tem os efeitos desestabilizadores e dissocializantes mais desastrosos para os que o sofrem. (CASTEL, pág 584, 2010) 
\end{citacao}

O que Castel relata em sua obra é comprovado em nossa pesquisa, porque existe esse medo do desemprego na prática e também essa realidade do desemprego que frequentemente nos aborrece.

O fato é que o desemprego acaba fazendo muitas pessoas, em especial as de menor renda familiar, se submeterem em trabalhos totalmente precários, tanto do ponto de vista do salário quanto da atividade realizada e mesmo assim não se importam com essa condição em que se encontram.

Os efeitos são realmente desestabilizadores, e é função do sistema capitalista para fazer a massa trabalhadora ser cada vez menos valorizada e procurar submeter-se a qualquer trabalho para ajudar a família ou até mesmo sustentar a própria família, seja qual for o motivo a função do sistema capitalista é usar seus mecanismos para transformar o ser humano em Mão de obra desvalorizada para aumentar seu objetivo final.

Não podemos mensurar os efeitos, mas eles existem e algumas provas deles estão nos relatos dos nossos entrevistados. Asseguir Castel nos diz:

\begin{citacao}
Mas o desemprego é a manifestação mais visível de uma transformação profunda da conjuntura do emprego. A precarização do trabalho constitui-lhe uma outra característica, menos espetacular põem ainda mais importante, sem dú-vida. (CASTEL, pág 514, 2010) 
\end{citacao}

Quando pensamos que o desemprego é a pior de todas as situações que podemos enfrentar em nossas vidas, percebemos que a situação da precariedade do trabalho é também de extrema importância porque independentes se estão falando de trabalho formal ou informal a precariedade existe em cada um deles. E como foi mostrado em nossa etnografia e em nossas entrevistas essa precarização está presente no cotidiano do trabalhador informal de Maracanaú. Segundo Castel:

\begin{citacao}
Mas a empresa falha igualmente em sua função integradora em relação aos jovens. Elevando o nível das qualificações exigidas para a admissão ela des-monetariza uma força de trabalho antes mesmo que tenha começado a servir. Assim, jovens que há vinte anos teriam sido intergrados sem problemas a pro-dução acham-se condenados a vagar de estágio em estágio ou de um pequeno serviço a outro. Porque a exigência de qualificação não corresponde sempre a imperativos técnicos. (CASTEL,pág 520 ,2010)
\end{citacao}

Castel nos relata na citação acima uma das problemáticas que percebemos ser ter maior importância para muitos trabalhadores informais de Maracanaú. Os jovens que trabalham na Ceasa informalmente geralmente procura oportunidade lá por não terem qualificação nenhuma para poder ingressar no mercado de trabalho formal, então como não servem para empresas porque não estão qualificados ou porque não possuem experiência esses jovens acabam encontrando no trabalho informal exercido na Ceasa e em outros lugares de Maracanaú a oportunidade que não encontraram no trabalho formal.

Através de toda nossa pesquisa e toda nossa leitura sobre o tema conseguimos concluir primeiramente que a exploração da mão de obra também existe no trabalho informal, as péssimas condições de trabalho, a baixa remuneração, a desvalorização do  trabalhador, trabalho infantil. Porém para entrar no trabalho informal não é necessário experiência e muito menos qualificação e é isso que o torna atrativo para quem precisa de uma primeira oportunidade de trabalho e geralmente não possui muita qualificação profissional ou nenhuma qualificação. 

O trabalho informal no município de Maracanaú representa uma válvula de escape para uma grande parte de trabalhadores em sua maioria jovens inexperientes.
De uma maneira bem geral podemos concluir também que ser empregado no trabalho informal é ser mal remunerado, trabalhar geralmente muito mais do que quarenta e quatro horas semanais e na sua maioria as atividades realizadas são bem exaustivas exigindo grande esforço físico e mental, porém é lá que muito adquirem experiência e responsabilidades como ter assiduidade e pontualidade e isso é um fator positivo que mostramos ao longo da pesquisa.

Para os empregadores do setor informal alguns também trabalham muitas horas porém não podemos dizer que sua remuneração é baixa, fazem as atividades mais tranquilas e gostam muito do trabalho exercido.

Portanto, de uma maneira geral os trabalhadores entrevistados consideram o trabalho informal ruim, devido a total exploração, mas não é isso que mais incomoda para esses trabalhadores o que mais incomoda é a baixa remuneração, porém ressaltam que é bem mais fácil conseguir um trabalho informal do que conseguir um trabalho formal. Para os empregadores do setor formal a maior dificuldade é a falta de tempo, porque sua atividade toma a maioria do seu tempo, mas consideram seu trabalho bem recompensatório do ponto de vista profissional e pessoal. 




































Portanto, nossa pesquisa nos mostra primeiramente a grande quantidade de 
significados contido dentro do conceito da palavra informalidade, mais 
precisamente dentro da palavra trabalho informal, existem muitas atividades que 
não conseguimos mensurar, porém numa regra geral entendemos como o trabalho sem 
carteira assinada, essa é definição do senso comum brasileiro quando tratamos do 
assunto trabalho informal. 

Esse senso comum não está totalmente errado, porém 
existem outros conceitos que se encaixam dentro da informalidade. O conceito 
trabalho informal ainda não é um conceito em que a maioria dos intelectuais 
concordem mas para a maioria dos brasileiros significa trabalhar sem carteira 
assinada.

Em nossas pesquisas conseguimos concluir que o trabalho informal para muitos 
habitantes de Maracanaú significa uma saída para melhorar sua condição humana e 
social, significa fazer parte de um determinado grupo social, isto é o grupo que 
participa ativamente da economia do município, que através de sua força de 
trabalho sustenta a família ou ajuda a família. 

E isso é de extrema importância 
para o bom convívio social e isso torna ainda mais importante o trabalho, mesmo 
que esse trabalho seja informal e extremamente precário. Mas não esquecendo o 
que \cite{augusta} nos diz ``Deve-se, portanto, recusar a ideia de que o 
trabalho informal se restringe às atividades de sobrevivência''. 

Para alguns dos habitantes de Maracanaú o trabalho informal significa mais do 
que apenas um trabalho qualquer, que te ajuda a ter uma condição mínima para 
sobreviver e conseguir suprir suas necessidades básicas com o mínimo de 
qualidade. Significa aquele trabalho extremante bom, o trabalho que vos realiza 
como pessoa, é aquilo que vos satisfaz e que não encontraríamos em nenhum outro 
lugar. 

Tanto do ponto de vista de uma boa condição de trabalho e de uma boa 
remuneração quanto do ponto de vista de uma realização pessoal em que através 
desse trabalho o conforto chegou até seu próprio lar de uma forma até mesmo 
inesperada. O trabalho informal lhes proporcionou algo que sua pouca formação 
acadêmica não iria lhe proporcionar em uma vida toda de trabalho formal.

Percebemos que para alguns esse trabalho informal tido por precário por diversos 
motivos já citados aqui, como o excesso de horas trabalhadas, condições ruins de 
trabalho, falta de folga remunerada etc. É considerado algo muito melhor do que 
diversos trabalhos formais, claro que estamos falando das pessoas que se 
tornaram proprietária de investimentos no setor informal, porque te proporciona 
uma boa remuneração, não paga tributos etc. 

Porém até chegar a esse patamar de 
tornar algo que para a maioria serve como saída para encontrar um trabalho que 
se torne em algo extremamente lucrativo e bom é necessário persistência, 
determinação, dentre outros adjetivos que percebemos serem precisos ao longo, 
principalmente do nosso trabalho de pesquisa. 

Quando adentramos ao mundo do trabalho informal podemos notar não apenas a 
importância do ponto de vista da remuneração quanto dá integração social que 
Maria Augusta nos diz:

\begin{citacao}
A reconfiguração da empresa e do emprego é portadora de questões teóricas 
fundamentais para a sociedade, pois sendo o trabalho uma relação social, não há 
como separar a sua função econômica do contexto social em que o mesmo se 
desenvolve. \cite{augusta}
\end{citacao}

Sem poder separar o trabalho do convívio social é cada vez mais clara a 
importância do trabalho na vida de qualquer individuo. Podemos, portanto notar 
que mesmo que seja informal nada diferencia com relação a essa integração social 
que o trabalho nos proporciona.

O trabalho informal em Maracanaú é e continua sendo algo que ajuda e facilita a 
vida de muitos habitantes e que como sabemos nem sempre se restringe a 
atividades de sobrevivência, alguns conseguem transformar em algo totalmente 
lucrativo e viver em boas condições de vida. Então percebe-se que nossa pesquisa 
mostrou na pratica o pensamento dos grandes intelectuais utilizados no trabalho 
em si, mostramos a importância do trabalho para essas pessoas que residem em 
Maracanaú, mostramos a definição conceitual de trabalho sem carteira assinada é 
relacionado com trabalho informal não só em Maracanaú mas também em todo o 
Brasil e mostramos na pratica o que Maria Augusta (2002)  nos diz a respeito do 
trabalho informal, que não é uma atividade que apenas o individuo faz para 
sobreviver, existem sim muitas pessoas em Maracanaú lucrando e usufruindo muito 
desse trabalho informal que para a grande maioria é sim muito precário, mas 
existe sim uma parcela de pessoas que conseguiram sobressair e se beneficiar 
muito através dele. 

Portanto, mesmo que seja precário informal e até mesmo ilegal. O trabalho 
informal é a fonte de renda e um dos poucos meio para que a população menos 
favorecida da sociedade. Mesmo que seja extremamente explorador no trabalho 
informal muitos moradores de Maracanaú conseguem melhorar de vida e realizar 
seus objetivos profissionais e pessoais, então concordamos com o Tavares e 
atestamos na prática que o esse trabalho não é apenas para sobrevivência. 
