\chapter{Conclusão}

Portanto, toda atividade de trabalho realizada por esses trabalhadores citados nessa pesquisa, transformam suas vidas para melhor, mesmo com todas as dificuldades enfrentadas diariamente. O trabalho realizado é exaustivo, porém percebemos que as pessoas que estão inseridas nele conseguem se satisfazer socialmente, e até mesmo economicamente, pois assim eles pode se sentir mais úteis e mais inseridos na sociedade por estarem dentro de alguma ocupação trabalhista.

Percebemos que na sociedade atual, ter uma ocupação de trabalho, além de ser um direito constitucional, é algo indispensável para satisfazer as necessidades basilares de qualquer cidadão. Notamos que o trabalho informal trouxe para a vida dessas pessoas (entrevistados da pesquisa) o que elas mais queriam, a satisfação de poder exercer alguma atividade de trabalho remunerado para ajudar na renda familiar. 

Por conseqüência, eles passaram a poder conquistar seus objetivos pessoais e obter experiência profissional como, por exemplo, adaptar-se ao cotidiano de trabalho e aprender a assumir as responsabilidades que são necessárias em qualquer profissão mesmo no caso do trabalho informal.

O trabalho informal proporcionou uma saída para muitos e de certa forma consegue compensar a falta de oportunidades e postos de trabalhos para todos.  Essa saída é muito comum para aqueles menos favorecidos, economicamente e intelectualmente, como foi mostrado no decorrer dessa pesquisa.

Como já foi relatado algumas vezes ao longo da pesquisa, o trabalho informal era considerado uma atividade a margem do sistema capitalista e era esperado que ao passar dos anos ele seria excluído. Porém o mesmo conseguiu e consegue se manter vivo até os dias atuais, ainda de forma precária, mas mesmo assim ele é de extrema importância para muitos moradores do município de Maracanaú pois assim ele consegue suprir uma lacuna de demanda por trabalho que existe para as pessoas menos qualificadas.

De fato constatamos que para os participantes de nossa pesquisa, a oportunidade de poder trabalhar informalmente ajuda bastante, pois ela dá a esses jovens e adultos uma oportunidade valiosa para crescerem profissionalmente.

Já demonstramos algumas das adversidades e complicações enfrentadas por essas pessoas que atuam no trabalho informal em Maracanaú. Notamos que desde o trabalhador menos ao mais favorecido, o trabalho informal é bastante valorizado. E mesmo consciente de sua grande exploração e desproteção social, consideram a atividade satisfatória porque para elas o que importa verdadeiramente é a remuneração que recebem em troca de sua força de trabalho.

 Basicamente, a maioria dos entrevistados não possuem nem mesmo o ensino médio completo e os jovens possuem pouca experiência profissional devido a grande dificuldade de encontrar o primeiro emprego. Eles declararam abertamente que precisavam de uma ocupação de trabalho para ajudar a família mesmo que sendo através do setor informal.
 
 Os jovens entrevistados que trabalham na CEASA, por exemplo, são homens em sua maioria pardos que possuem uma idade média entre dezesseis a vinte e um anos e atuam principalmente carregando e descarregando as frutas que chegam a bordo dos caminhões de carga, recebem uma remuneração mensal, dependendo do local onde trabalham, de aproximadamente oitocentos reais se trabalharem todos os dias do mês ou uma diária que varia entre trinta a cinqüenta reais para aqueles que trabalham apenas um ou dois dias na semana. 
 
No caso específico da Sra. Dina, que é proprietária de uma barraquinha de churrasco, a situação é um pouco mais tranqüila, ela não trabalha nos horários em que o clima está mais quente e sua atividade é mais no preparo das carnes além de receber o pagamento pela venda de seus produtos. Segundo ela, sua atividade lhe remunera um valor que varia entre cento e cinqüenta até duzentos e cinqüenta reais diariamente.

Para os ajudantes de Sra. Dina a remuneração é um pouco menor do que os jovens que trabalham na CEASA. Porém, eles acham importante trabalhar na barraca dela, pois com essa remuneração, ajudam a complementar a renda em casa e conseguem comprar produtos de uso pessoal.

Com tudo que já foi exposto, fica clara a importância do trabalho informal no município de Maracanaú. Uma vez que a sociedade atual é capaz de excluir diversas pessoas de um ambiente de trabalho favorável, por conta da pouca experiência e desqualificação, ou mesmo preconceitos. O trabalho informal consegue absorver essa força de trabalho, inserindo assim os menos favorecidos e fazendo com que eles consigam uma vida mais digna para si e para seus familiares.  
