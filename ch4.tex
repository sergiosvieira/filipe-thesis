\chapter{O Cotidiano dos Trabalhadores Informais de Maracanaú}

 Neste capitulo abordaremos o assunto a partir de uma etnografia e algumas entrevistas. Escolhemos 
 como objeto principal para serem entrevistados os trabalhadores informais da Ceasa (Central de 
 abastecimento do Ceará), entrevististamos alguns trabalhadores e ex-trabalhadores informais da 
 Ceasa. Procuramos entrevistar aqueles que estavam trabalhando na total informalidade e com funções
 totalmente degradantes de trabalho.
 
 \section{O depoimento dos trabalhadores informais sobre o cotidiano de trabalho na Ceasa}
 
 Nosso primeiro entrevistado se chama Jefferson Duarte, masculino, pardo, 21, solteiro, ensino médio 
 incompleto, não tem filho e ainda mora com os pais, Jefferson nos disse que trabalhou na Ceasa 
 durante três anos, onde se vendiam uvas. Trabalhava de Segunda-feira até Sábado e alguns Domingos.
 Existia uma divisão do trabalho conforme os dias da semana, dia de segunda-feira e quinta-feira 
 como é dia de chegar carregamento e de distribuir a mercadoria para os clientes a carga horária de 
 trabalho se tornava maior, começava as duas horas da manhã até as 4 da tarde, totalizando quatorze
 horas de trabalho diário. Terça-feira, Quarta-feira, Sexta-feira e Sábado o horário era de quatro
 da manhã até quatro da tarde. O serviço é basicamente o mesmo, porém com uma carga horária inferior
 aos dias anteriormente citados, então o que se fazia era organizar a mercadoria e pegar o caminhão
 e distribuir para os clientes. E depois que chegava as quatro da manhã só iam comer as nove horas
 da manhã. Cada trabalhador ganhava quinze reais de almoço e cinco reais de merenda, e não existia 
 uma hora certa nem para almoçar nem para merendar cada um comia quando o trabalho aliviava um pouco 
 e sempre eram de dois em dois para comer (isso em uma equipe de seis trabalhadores), os outros 
 continuavam trabalhando. Sendo que de remuneração cada um recebia duzentos reais por semana. E 
 sendo que não existia muito critério para contratação de novos funcionários se algum amigo te 
 levasse e estivesse precisando de alguém em tal lugar era só se apresentar e o dono já te mandava 
 trabalhar, sem nem mesmo saber idade, qualificação etc.

 No aspecto relacionado as dificuldades do trabalho Jefferson nos relatou que a desempenhava várias
 funções ao mesmo tempo, já sofreu acidente no trabalho, não tinha folga remunerada, não tinha
 férias, e se faltar mesmo com atestado a remuneração semanal era descontada, porque o patrão 
 achava que os atestados eram falsos. A relação com o patrão era uma relação complicada se errasse
 em algum serviço o trabalhador levava logo o nome de burro e outras palavras de baixo calão. 
 As maiores dificuldades enfrentadas segundo este depoimento de Jefferson "é porque todo trabalho 
 é braçal, além de descarregar as mercadorias no obro ainda tinha que levar para outro lugar em
 um carrinho carregando sozinho." além de tudo ser manual cada trabalhador realizava várias atividades
 dentre as quais: descarregar o caminhão, pesar as frutas, selecionar as frutas boas e ruins e depois 
 colocar todas as caixas secas de volta no caminhão.

 Ao longo desses três anos Jefferson passou por inúmeras dificuldades relacionadas a sua situação no
 trabalho ele nos relatou um acidente de trabalho: "Eu estava levando um carregamento que tinha quinze 
 caixas no carrinho mas só cabiam dez caixas, mesmo assim o patrão me obrigou a levar.. no caminho eu 
 enganchei minha mão no carregamento e acabei tendo um corte feio. Depois que isso aconteceu o meu 
 patrão só fez jogar um pouco de água gelada na minha mão e colocar uma atadura e depois disse que já
 podia voltar para o trabalho. Nem fui para o hospital e nem fui indenizado". Outra dificuldade relatada
 foi que no final de ano como existem um movimento maior, Jefferson acabou saindo de casa para trabalhar
 as duas horas da madrugada e chegando em casa apenas as dez horas da noite.

 Outro problema que Jefferson nos relata do trabalho era a grande quantidade de insetos e animais que 
 contaminavam o ambiente o que se tornava mais fácil de ficar doente. Tinha muita lama e passava vários
 animais que com certeza contaminava tudo. Jefferson nos diz: “E também mais um problema era que a 
 grande quantidade de trabalhadores com idade inferior a dezoito anos e que a fiscalização era uma vez
 perdida e quando chegavam alguns dos trabalhadores de menor se escondiam, e os outros a fiscalização
 só perguntava idade sem pedir documento nenhum”. Outro fato importante é que dentre os trabalhadores 
 existia um vocabulário comum quando um o ou alguns trabalhadores desviavam a carga para si, o que para 
 o a maioria do senso comum é chamado de roubo ou furto, para eles é chamado galinha morta, gol, golaço.
 
 Jefferson tinha uma real noção de que seu trabalho era um trabalho extremamente precário, no entanto
 a facilidade com que se conseguia uma vaga de trabalho fazia com que ele acabasse sempre indo para
 Ceasa quando precisava de dinheiro. Em suas palavras Jefferson diz: “A situação era precária até demais,
 se sentia explorado sim, mas não queria nem saber, o importante era estar com dinheiro no bolso sábado
 (dia do pagamento)”.

 Apesar de Duarte admitir que o trabalho era explorador e poucos aguentavam ficar ele continuava 
 trabalhando, porém um dia ele sentiu um problema de saúde e foi até ao médico. Ele relatou esse fato
 assim: “Um dia passei mal e fui para o médico e ele me perguntou qual era meu trabalho. eu disse que 
 trabalhava na Ceasa expliquei direitinho como era então ele me disse que eu estava trabalhando muito 
 e descansando pouco e se continuasse assim só teria mais seis meses de vida, então eu optei por sair 
 da Ceasa e nesse momento foi que eu cai na real e vi que não era aquilo que eu queria para o resto da 
 minha vida então voltei a estudar". Portanto, percebe-se que mesmo quando alguém não se importa em ser
 explorado o corpo não consegue aguentar.

 Entrevistei outro ex-trabalhador informal da Ceasa: Cristiano Martins, solteiro, vinte dois anos, 
 pardo, ensino médio completo e não tem nenhum filho. Ele trabalhou na Ceasa durante um ano, quando 
 entrou tinha apenas dezenove anos e resolveu trabalhar lá porque precisava de dinheiro e não tinha 
 oportunidade no mercado formal e também porque não tinha experiência comprovada na carteira de trabalho
 e possuir pouca qualificação e capacitação que pudesse lhe proporcionar uma boa ocupação no mercado de
 trabalho e como a demanda por mão de obra e a facilidade para conseguir um trabalho informal na Ceasa
 é grande alguns amigos o convidaram a ir trabalhar lá e ele acabou aceitando.

 
 Cristiano descreve o trabalho que ele realizava na Ceasa como um trabalho extremamente difícil por 
 alguns motivos como acordar cedo, carregar muito peso e ser extremamente explorado, além de receber 
 pouco. Como Cristiano só trabalhava nos dias de terça-feira e quinta-feira só ganhava uma diária 
 correspondente no valor de trinta e cinco reais (isso porque era no setor que vendia predominantemente 
 morango, o preço da diária varia em cada local de trabalho), o horário de trabalho era de duas da manhã 
 até ao meio-dia, no entanto eventualmente ele entrava as nove da noite e só saia ao meio-dia (mesmo 
 assim continuava ganhando o mesmo valor da remuneração diária dos outros dias). A principal dificuldade
 que ele encontrou foi de acostumar com o horário de trabalho.

 No seu cotidiano de trabalho Cristiano Martins desenvolvia diversas atividades, desde limpar o caminhão
 até organizar e vender toda a mercadoria comercializada, a atividade trabalhista se tornava pesada porque
 a atividade desenvolvida era descarregar as caixas com morango e levar para outros compartimentos onde a
 mercadoria era armazenada ou para ser vendida, no entanto mesmo tendo como função principal descarregar 
 a mercadoria Cristiano nos diz que desenvolvia qualquer atividade que fosse pedida. Com toda essa exploração
 e com tantas dificuldades um ponto positivo foi que Cristiano nunca sofreu nenhum acidente no trabalho, 
 o que é algo frequente entre os trabalhadores informais da Ceasa que trabalham descarregando os caminhões
 com frutas e outros alimentos. A seguir Cristiano nos relata que passou por uma situação complicada no 
 trabalho: ``Teve um dia que eu estava organizando as mercadorias e não tinha nenhum carrinho para levar
 para os outros compartimentos, então o encarregado me pediu para levar uma caixa na cabeça que pesava 
 aproximadamente uns cinquenta quilos, mas eu disse que não ia levar, até porque não tinha condição por 
 causa do meu porte físico, eu achei um total absurdo''.

 Cristiano se sentia totalmente explorado no trabalho que realizava, principalmente porque ele trabalhava
 muito e ganhava pouco, não era respeitado, não era valorizado de nenhuma maneira. Mas como ele precisava 
 muito do dinheiro ia somente por isso e com essa diária que ele ganhava custeava suas necessidades pessoais.

 Na Ceasa trabalham diversas pessoas, algumas com intenções de ganhar a dinheiro trabalhando honestamente
 e outros nem tanto, sabendo que é comum ocorrer alguns desvios e furto de mercadoria na Ceasa o entrevistado 
 nos diz que aonde ele trabalhou acontecia sim esse tipo de coisa, mas que não são todos os trabalhadores que
 costumam fazer isso e também depende de cada lugar, cada local tem sua particularidade para esses desvios,
 se o dono, por exemplo, não controla sua mercadoria fica mais fácil desviar, mas se ele controla existe 
 outras maneiras de burlar o controle do patrão. Segundo o depoimento de Cristiano isso ocorria no seu setor
 de trabalho: “Quando eu e o funcionário mais antigo fazíamos entrega no caminhão para outros locais fora da
 Ceasa o funcionário mais antigo, que era o responsável por toda mercadoria, acabava desviando algumas caixas,
 porque o dono do morango não tinha um controle rigoroso da mercadoria e isso facilitava o desvio”. Depois 
 desse depoimento entendemos que cada setor tem suas particularidades com relação ao desvio de carga, porém 
 não existe somente esse tipo de pessoa trabalhando na Ceasa, existem também trabalhadores honestos que 
 procuram melhorar de vida através dessa ocupação de trabalho.

 Os dois próximos entrevistados da pesquisa ainda continuam trabalhando na Ceasa. O primeiro deles é Alexandre
 Rabelo, pardo, vinte e um anos, solteiro, ensino médio incompleto. Alexandre começou a trabalhar na Ceasa com 
 dezenove anos e o que mais o motivou para ir trabalhar lá foi à falta de outras oportunidades de emprego e 
 também a necessidade por um trabalho rápido. Os seus amigos lhe motivaram a ir trabalhar na Ceasa e ele acabou 
 aceitando. Alexandre trabalha de segunda-feira até sábado, com dois horários diferentes nos dias de terça-feira
 e quinta-feira ele entra às onze e meia da noite e sai as nove e meia da manhã. E nos outros dias que ele trabalha 
 o horário é de duas e meia da madrugada até as dez da manhã, seu ganho semanal é de duzentos e setenta e cinco reais. 
 Sua única folga é no domingo e teve suas primeiras férias depois de um ano de trabalho

 O trabalho desenvolvido por Alexandre na Ceasa é de separar as frutas boas das 
frutas ruins e ele considera boa sua atividade de trabalho só acha ruim o fato 
de não ter tempo específico para merendar, é só merendar rapidamente e já voltar 
para o serviço e também quando falta algum outro funcionário, isso lhe faz 
trabalhar mais para compensar a ausência do seu colega de trabalho e é nesse 
momento que ele se sente explorado no trabalho. A única dificuldade encontrada 
por Alexandre foi se acostumar com o horário de trabalho e o trajeto 
casa-trabalho que lhe deixa inseguro devido ao fato de terem poucas pessoas no 
horário que ele sai para trabalhar, fora isso ele nos relata que o resto foi 
fácil.

 Com relação à experiência adquirida no trabalho ele nos diz que foi de extrema 
importância para sua vida porque o tornou mais responsável e lhe ajudou a 
conquistar muitos dos seus objetivos pessoais, além de poder ajudar sua mãe 
financeiramente, que era uma motivação particular de Alexandre.

 O segundo entrevistado é David Martins, pardo, dezoito anos, masculino, 
solteiro, ensino médio incompleto. Trabalha na Ceasa desde os dezesseis anos de 
idade. David optou por ir trabalhar na Ceasa porque estava precisando ajudar a 
família e precisava de um trabalho rápido, então seus amigos indicaram a Ceasa 
como um local que com facilidade se encontrava um emprego porque lá sempre 
precisa de mão de obra. Com a influência dos amigos David conseguiu o trabalho e 
começou rapidamente, já na segunda semana de trabalho estava adaptado ao 
serviço. Os dias de trabalho de David são as terças-feiras e quintas-feiras, com 
horário de entrada a uma hora da madrugada e horário de saída as dez horas da 
manhã, com uma remuneração diária de cinquenta reais, e cinco reais para custear 
sua merenda.

 Segundo o relato de David seu trabalho é carregar e descarregar os caminhões 
com as caixas de frutas e fazer as entregas, dentre outras atividades que sejam 
necessárias para o melhor desenvolvimento do serviço. Sua principal dificuldade 
no começo foi acostumar-se com o horário de trabalho, porque nunca tinha 
trabalhado em nenhum horário na verdade, mas com o passar dos anos isso se 
tornou algo normal e hoje não existe mais essa dificuldade. Além do horário 
David também passou por um momento complicado no começo do trabalho quando 
sofreu um acidente quando estava descarregando o caminhão com outro colega de 
trabalho e por um descuido seu colega acabou jogando uma caixa na sua cabeça que 
pesa aproximadamente uns quarenta quilos, depois desse ocorrido seu patrão só 
perguntou se ele queria ir para casa e não ofereceu nenhuma ajuda para leva-lo 
ao médico ou sequer o indenizou por esse acidente sofrido no serviço. Passando 
essas dificuldades no começo hoje David se considera totalmente adaptado ao 
serviço. David nos diz também que sua relação com os colegas de trabalho é boa 
e também com seu patrão o ambiente é sempre de descontração. Quando pergunto se 
ele se sente explorado no seu trabalho ele responde que sim e diz “Não quero 
mais isso para mim e nem para ninguém, porque além do trabalho ser muito ruim o 
pior de tudo foi que atrapalhou meus estudos e eu parei no primeiro grau do 
ensino médio”. 

 Com relação às boas coisas que o trabalho lhe trouxe ele só consegue citar o 
dinheiro, que serve muito para ajudar sua família e custear seus interesses 
pessoais como comprar roupa e utilizar o dinheiro para sair com os amigos.

 O que conseguimos notar em relação a esses trabalhadores da Ceasa que foram 
entrevistados é que existe uma necessidade de ajudar a família e de se auto 
ajudar, então eles acabam aceitando o primeiro trabalho que aparece, sem se 
preocupar com as consequências futuras como, por exemplo, deixar os estudos para 
segundo plano. Mas a própria pressão familiar por uma ajuda financeira os faz 
optar por esse caminho. O fato é que a atividade de trabalho realizada é 
verdadeiramente o que menos importa para esses trabalhadores eles só se 
preocupam com a remuneração, mesmo que tenham que sofrer qualquer tipo de 
adversidade por parte da atividade exercida.

 Entrevistei outra pessoa que trabalha informalmente conhecida popularmente como 
Dona Dina,ela possui uma barraca de churrasco,que fica localizada no município 
de Maracanaú mais precisamente no bairro da Pajuçara . Dina como é popularmente 
conhecida tem cinquenta e sete anos de idade, divorciada, três filhos, é de cor 
parda, e só concluiu até a sétima série do ensino fundamental (fundamental 
incompleto). Ela nos relata um pouco da história de como tudo começou para hoje 
ela ser uma trabalhadora informal que consegue ganhar a vida e de certa maneira 
ter muitos benefícios com sua profissão. Tudo começou quando por acaso, o 
espetinho e o ponto já tinham um dono e esse dono era seu amigo, certo dia ele 
adoeceu e devido a sua doença ser um pouco grave, segundo dona Dina, ele 
ofereceu a barraca e o ponto para por cinco mil reais e disse que ela poderia 
pagar de qualquer forma, então ela aceitou a proposta e começou a tocar o 
investimento passou um ano trabalhando praticamente só e depois de um ano de 
muita dificuldade contratou um funcionário para ajuda-la (ela paga cem reais por 
semana para ele),passado essas dificuldades hoje ela vende em média cento e 
cinquenta espetinhos por dia, fora os refrigerantes, cerveja, cachaça e baião 
nos dia de sexta-feira. Ela não se queixa do trabalho a única adversidade para 
ela é que uma parte do trabalho (comercialização das mercadorias) é realizada no 
meio da rua, mas ressalta a importância do investimento para sua vida e descreve 
com algo essencial para sua existência pessoal e profissional. Há pouco tempo 
dona Dina contratou outro funcionário com a mesma remuneração do seu primeiro 
funcionário e ela relata que sua relação com os funcionários é boa, a única 
cobrança é com relação a higiene dos funcionários, ela dá folga apenas no 
domingo e no final do ano dá vinte dias de férias no mês de dezembro. E já faz 
três anos que ela está desenvolvendo está atividade e se considera muito 
satisfeita e realizada por ter essa profissão.

\section{Etnografia realizada sobre o cotidiano do trabalho informal}

No dia vinte e seis de junho de dois mil e quatorze realizei minha pesquisa 
etnográfica com dona Dina e seus dois funcionários. Cheguei à sua residência as 
sete e meia, horário em que ela e seus ajudantes começaram a cortar as carnes, 
fazer os temperos e preparar os espetos para serem comercializados mais tarde. 
Tudo é realizado em um compartimento especifico dentro de sua própria casa, que 
é onde todas as preparações dos alimentos ocorrem e também onde as carnes são 
separadas e onde estão os vários utensílios que são usados no serviço, mas antes 
de eu te chegado dona dina e seu marido já tinham ido até a Ceasa as quatro 
horas da manha comprar as carnes que serão preparadas para a comercialização. 

Todo procedimento é feito por Dina e seus ajudantes, ela primeiramente lava toda 
a carne e depois começa a cortar (serviço demorado e um pouco arriscado devido 
ao fato de que qualquer erro pode significar a perca de um dedo ou vários dedos 
da mão) toda carne é separa em varias vasilhas de plástico numa mesa separada 
apenas para realização desse serviço, separando cada carne no seu devido lugar e 
seus ajudantes vão colocando as carnes nos palitos de espeto. Com muita técnica 
eles colocam a carne no palito, mas mesmo com muito cuidado às vezes acabam 
furando os dedos em algum momento de descuido. Esse procedimento de corte e 
ajuste da carne acontece entre as sete e meia da manhã até as onze e meia da 
manhã. Depois desse procedimento realizado de manhã dona dina e seus ajudantes 
vão almoçar. Os ajudantes almoçam em suas respectivas casas e depois retornam ao 
serviço.

Os ajudantes voltam ao serviço às uma e meia da tarde e começam a organizar os 
objetos que serão levados até o ponto de comercio, que localiza-se 
aproximadamente uns trinta a quarenta metros da casa de Dina, em um terreno bem 
próximo da sua casa, começa a organização das mesas e cadeiras, copos, vasilhas 
com os espetos crus, copos descartáveis, sacolas, balde de água que serve para 
lavar as mãos tanto dos ajudantes quanto dos clientes, a churrasqueira, o carvão 
e vários pedaços de madeira e papelão, uma lona que serve para proteção contra o 
sol e as demais matérias que serão comercializados entre eles cachaça e 
refrigerante. 

Aproximadamente as duas horas da tarde os ajudantes saem para começar o trabalho 
no ponto de comércio e dona dina fica em casa descansando ou fazendo outras 
atividades que julgue ser necessária. Quando chegam, os ajudantes primeiramente 
montam a barraca e colocam as mesas e cadeiras por perto, mas ainda não as deixa 
organizada, depois os ajudantes montam a churrasqueira, colocam um cesto de lixo 
no chão e o balde de água em uma mesa separada. Ao término desse procedimento de 
ajustes os ajudantes colocam o carvão na churrasqueira e misturam com os pedaços 
de madeira e papelão para ascender o fogo. Eles têm um pouco de dificuldade para 
ascender o fogo por causa dos ventos, mas conseguem ao fazer esse procedimento 
eles me relatam que o ruim é a fumaça que faz e que a temperatura se eleva ainda 
mais. Depois de ascender o fogo eles começam a assar as carnes de quatorze em 
quatorze (que é o que cabe na churrasqueira) vão assando até terminarem tudo, 
esse procedimento dura em média duas horas ou duas horas e meia, enquanto isso a 
clientela é fraca, aparece poucas pessoas, uns para beber uma dose de cachaça e 
outros para comer um espetinho. 

Quando é aproximadamente umas cinco horas dona Dina chega e pede para os 
ajudantes organizarem as mesas e cadeiras, pouco tempo depois a clientela começa 
a surgir em maior número e em um descuido dona dina é enganada por um cliente 
que pede o espeto, come, porém sai de sem pagar, essa eventualidade acontece 
segundo ela e conseqüentemente isso lhe deixa bastante irritada, mas o serviço 
continua e a cliente chega de vários locais e de diversos meios de condução, a 
pé, de carro, de motos, de bicicleta acompanhada por outras pessoas ou não. 
Quando o movimento aumenta quem atendente quase toda a clientela é os ajudantes, 
dona Dina passa a maior parte do tempo conversando com alguns clientes ou 
amigos, uma vez ou outra ela atende algum cliente depois volta a ficar sentada 
conversando. Quando é aproximadamente umas seis e meia da noite o movimento 
chega ao auge, todas as mesas estão lotadas e várias pessoas pedem os diversos 
produtos oferecidos, isso permanece até umas sete e meia da noite, depois desse 
horário aparecem outros clientes, mas o movimento diminui bastante, mas o 
trabalho continua até às oito e meia da noite depois desse horário todo os 
materiais são recolhidos pelos ajudantes de dona Dina, primeiro eles começam a 
levar a cadeira, arrumam as cadeiras e levam nos braços mesmo, depois levam 
também as mesas, algumas são levadas em um carrinho de mão e outras são levadas 
nos braços de um dos ajudantes, quando os dois ajudantes terminam essa tarefa 
eles começam a organizar os materiais e guardam tudo em um compartimento dentro 
do carrinho de espeto, derrama-se a água suja que foi utilizada e também a água 
que restou e guardam o balde. 

Quando tudo está organizado os dois ajudantes levam os carrinhos de volta para 
casa de dona Dina, que fica bem próximo, e colocam dentro da sua garagem (lugar 
em que ficam os dois carrinhos do espeto, carvão e a churrasqueira), depois que 
tudo está na casa os espetinhos que sobram são colocados no freezer e segundo 
Dina ainda podem ser aproveitados até oito dias depois. É descarregada toda a 
mercadoria contida dentro do carrinho e colocada no compartimento da casa que 
fica os materiais específicos para o trabalho, para mais tarde serem lavados por 
dina e seu marido, depois de descarregarem as mercadorias os ajudantes de dona 
dina estão liberado do serviço e volta para suas casas, isso aproximadamente 
umas nove horas da noite para recomeçar novamente a jornada de trabalho às sete 
e meia da manhã do outro dia, depois que os funcionários se vão, Dina vai contar 
o dinheiro que foi ganho no dia trabalhado, que nesse dia deu cento e noventa e 
sete reais porque foram vendidos quarenta espetinhos que possuem um preço de 
dois reais e cinqüenta centavos e quinze espetinhos que valem três reais, dez 
refrigerantes de latinha foram vendidos a dois reais, doze doses de cachaça 
vendidas a um real e outras dez latinhas de cerveja vendidas a dois reais. 
Depois da contagem do dinheiro apurado o meu próprio expediente de trabalho 
etnográfico já está encerrado, isso aproximadamente umas nove e meia da noite.

Através de toda nossa pesquisa empírica podemos resaltar na pratica algumas 
questões que Robert Castel nos relata em sua obra intitulada de: A questão da 
metamorfose social.

\begin{citacao}
Assim o desemprego é seguramente,hoje, o risco social mais grave, o que tem os 
efeitos desestabilizadores e dissocializantes mais desastrosos para os que o 
sofrem. Castel,2010, pág 584. 
\end{citacao}

O que Castel relata em sua obra é comprovado em nossa pesquisa, porque existe 
esse medo do desemprego na prática e também essa realidade do desemprego que 
frequentemente nos aborrece. O fato é que o desemprego acaba fazendo muitas 
pessoas, em especial as de menor renda familiar, se submeterem em trabalhos 
totalmente precários, tanto do ponto de vista do salário quanto da atividade 
realizada e mesmo assim não se importam com essa condição em que se encontram.

Os efeitos são realmente desestabilizadores, e é função do sistema capitalista 
para fazer a massa trabalhadora ser cada vez menos valorizada e procurar 
submeter-se a qualquer trabalho para ajudar a família ou até mesmo sustentar a 
própria família, seja qual for o motivo a função do sistema capitalista é usar 
seus mecanismos para transformar o ser humano em Mao de obra desvalorizada para 
aumentar seu objetivo final. Não podemos mensurar os efeitos, mas eles existem e 
algumas provas deles estão nos relatos dos nossos entrevistados. Asseguir Castel 
nos diz:

\begin{citacao}
Mas o desemprego é a manifestação mais visível de uma transformação profunda da 
conjuntura do emprego. A precarização do trabalho constitui-lhe uma outra 
característica, menos espetacular põem ainda mais importante, sem dúvida Castel, 
2010, pág 514.
\end{citacao}

Quando pensamos que o desemprego é a pior de todas as situações que podemos 
enfrentar em nossas vidas, percebemos que a situação da precariedade do trabalho 
é também de extrema importância porque independentes se estão falando de 
trabalho formal ou informal a precariedade existe em cada um deles. E como foi 
mostrado em nossa etnografia e em nossas entrevistas essa precarização está 
presente no cotidiano do trabalhador informal de Maracanaú. Segundo Castel:

\begin{citacao}
Mas a empresa falha igualmente em sua função integradora em relação aos jovens. 
Elevando o nível das qualificações exigidas para a admissão ela desmonetariza 
uma força de trabalho antes mesmo que tenha começado a servir. Assim, jovens que 
há vinte anos teriam sido intergrados sem problemas a produção acham-se 
condenados a vagar de estágio em estágio ou de um pequeno serviço a outro. 
Porque a exigência de qualificação não corresponde sempre a imperativos 
técnicos, Castel, 2010, pág 520. 
\end{citacao}

Castel nos relata na citação acima uma das problemáticas que percebemos ser ter 
maior importância para muitos trabalhadores informais de Maracanaú. Os jovens 
que trabalham na Ceasa informalmente geralmente procura oportunidade lá por não 
terem qualificação nenhuma para poder ingressar no mercado de trabalho formal, 
então como não servem para empresas porque não estão qualificados ou porque não 
possuem experiência esses jovens acabam encontrando no trabalho informal 
exercido na Ceasa e em outros lugares de Maracanaú a oportunidade que não 
encontraram no trabalho formal.