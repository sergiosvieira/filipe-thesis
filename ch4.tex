\chapter{O Cotidiano dos Trabalhadores Informais de Maracanaú}

Neste capítulo abordaremos o assunto a partir de uma etnografia e algumas entrevistas. Escolhemos como objeto principal de pesquisa os trabalhadores e ex-trabalhadores informais da CEASA (Central de abastecimento do Ceará) \footnote{Grande polo comercial do governo do estado do Ceará, onde funciona principalmente o comercio de produtos hortifruticultura em atacado, para donos de grandes comércios e também de pequenos comércios.}. Procuramos entrevistar aqueles que estavam trabalhando na total informalidade e com funções totalmente degradantes de trabalho. Além das entrevistas, fizemos uma etnografia com a Sra. Dina, proprietária de uma barraca que vende churrasco, que foi de grande importância para compreensão desta pesquisa. 

Nosso primeiro entrevistado chama-se Jefferson Duarte, masculino, pardo, vinte e um anos de idade, solteiro, ensino médio incompleto, não tem filho e ainda mora com os pais, Jefferson nos disse que trabalhou na CEASA durante três anos em um espaço onde eram vendidas uvas. Antes de começar a trabalhar, somente fazia o ensino médio. 
 
Ele trabalhava de segunda-feira a sábado e as vezes trabalhava aos domingos. Existia uma divisão do trabalho conforme os dias da semana. Na segunda-feira e na quinta-feira, como é dia de chegar carregamento e de distribuir a mercadoria para os clientes, a carga horária de trabalho se tornava maior, começava as duas horas da manhã e se estendia até as quatro da tarde, totalizando quatorze horas de trabalho diário.
 
Nas terças, quartas, sextas e sábados o horário era de quatro da manhã até quatro da tarde. O serviço era basicamente o mesmo, porém com uma carga horária inferior aos dias anteriormente citados, então o que se fazia era organizar a mercadoria, pegar um caminhão e distribuir o produto para os clientes. %E depois que chegava as quatro da manhã só iam comer as nove horas da manhã. 
 
Cada trabalhador ganhava quinze reais de almoço e cinco reais de merenda e não existia uma hora certa nem para almoçar nem para merendar. Só podiam comer quando o trabalho aliviava um pouco e sempre eram de dois em dois (isso em uma equipe de seis trabalhadores). Cada trabalhador recebia duzentos reais por semana. Como não existiam muitos critérios para contratação de novos funcionários, a contratação da mão-de-obra era feita quando surgisse uma oportunidade, era só se apresentar ao dono da venda e já te mandavam trabalhar, sem nem mesmo saber sua idade ou sua qualificação.

No aspecto relacionado as dificuldades do trabalho, Jefferson nos relatou que desempenhava várias funções ao mesmo tempo, já sofreu acidente no trabalho, não tinha folga remunerada, não tinha férias, e se faltar mesmo com atestado a remuneração semanal era descontada, porque o patrão sempre achava que os atestados eram falsos.
 
A relação com o patrão era complicada, se errasse em algum serviço o trabalhador levava logo o nome de burro e outras palavras de baixo calão. As maiores dificuldades enfrentadas, segundo Jefferson ``é porque todo trabalho é braçal, além de descarregar as mercadorias no obro ainda tinha que levar para outro lugar em um carrinho carregando sozinho.''. Além de tudo feito de forma braçal, cada trabalhador tinha que realizar várias atividades, como: descarregar o caminhão, pesar as frutas, selecionar as frutas boas e ruins e depois colocar todas as caixas secas de volta no caminhão. 
 
 Ao longo desses três anos Jefferson passou por inúmeras dificuldades relacionadas a sua situação no trabalho. Ele nos relatou um acidente de trabalho: 
 
 \begin{citacao}
Eu estava levando um carregamento que tinha quinze caixas no carrinho mas só cabiam dez caixas, mesmo assim o patrão me obrigou a levar. No caminho eu enganchei minha mão no carregamento e acabei tendo um corte feio. Depois que isso aconteceu o meu patrão só fez jogar um pouco de água gelada na minha mão e colocar uma atadura e depois disse que já podia voltar para o trabalho. Nem fui para o hospital e nem fui indenizado.  
 \end{citacao}

Aqui conseguimos entender um pouco como é complicada a vida de alguém que trabalha informalmente na CEASA, o próprio entrevistado nos relatou que achou isso um total absurdo e o pior não foi o acidente em si, foi à falta de apoio do seu patrão. Então a partir desse primeiro depoimento feito, nota-se como o trabalhador é totalmente desvalorizado. O pior de tudo é que para Jefferson, isso é o que menos importa, o importante é a remuneração adquirida no trabalho, sem se preocupar minimamente com sua saúde e integridade física, neste ponto percebemos algo bastante negativo dentro do trabalho informal.

Outra dificuldade relatada foi que no final de ano, como existe mais trabalho a ser realizado, Jefferson acabou saindo de casa para trabalhar às duas horas da madrugada e chegando em casa apenas as dez horas da noite.

Outro problema que Jefferson nos relatou era a grande quantidade de insetos e animais que contaminavam o ambiente, fazendo com que fosse mais fácil contrair uma doença. Havia muita lama em seu local de trabalho e passavam vários animais que acabavam por contribuir com uma maior contaminação do local. Jefferson nos diz:

\begin{citacao}
E também mais um problema era que a grande quantidade de trabalhadores com idade inferior a dezoito anos e que a fiscalização era uma vez perdida e quando chegavam alguns dos trabalhadores de menor se escondiam, e os outros, a fiscalização só perguntava a idade sem pedir documento nenhum.
\end{citacao}

Neste depoimento Jefferson acha até engraçado o fato da fiscalização não conseguir identificar quem é de menor e quem não é porque as ``crianças'' que trabalham na CEASA tem uma massa corporal bem desenvolvida por serem acostumados a carregar muito peso e como não é exigido nenhum documento de identificação nas abordagens, fica complicado de saber quem é de menor e quem não é.

Outro fato é que dentre os trabalhadores existia um vocabulário comum quando alguns trabalhadores desviavam a carga para si, o que legalmente conhecido como roubo, para eles é chamado ``galinha morta'', ``gol'', ``golaço''. Apesar de achar o roubo e desvio de carga algo totalmente normal, o entrevistado ressalta que as os trabalhadores que fazem essa escolha possuem total liberdade de trabalhar apenas honestamente, porém sente-se seduzidos pelas facilidades e acabam desviando carga para ganhar um dinheiro desonesto por fora.

Jefferson tinha uma real noção de que seu trabalho era um trabalho extremamente precário, no entanto a facilidade com que se conseguia uma vaga fazia com que ele acabasse sempre indo para a CEASA quando precisava de dinheiro. Em suas palavras Jefferson diz: ``A situação era precária até demais, se sentia explorado sim, mas não queria nem saber, o importante era estar com dinheiro no bolso sábado (dia do pagamento)''. 

Aqui conseguimos compreender e ter uma pequena noção do que representa o trabalho informal para Jefferson, apesar de admitir a exploração demasiada ele não considerava esse aspecto tão ruim, porque o que importava apenas era a remuneração.

Apesar de Jefferson admitir que o trabalho era explorador e poucos aguentavam ficar, ele continuava trabalhando, porém um dia ele sentiu um problema de saúde e foi até ao médico. Ele relatou esse fato assim:

\begin{citacao}
Um dia passei mal e fui para o médico e ele me perguntou qual era meu trabalho. Eu disse que trabalhava na CEASA, expliquei direitinho como era, então ele me disse que eu estava trabalhando muito e descansando pouco e se continuasse assim só teria mais seis meses de vida, então eu optei por sair da CEASA e nesse momento foi que eu cai na real e vi que não era aquilo que eu queria para o resto da minha vida, então voltei a estudar. 
\end{citacao}

Entrevistei outro ex-trabalhador informal da CEASA: Cristiano Martins, solteiro, vinte dois anos, pardo, ensino médio completo e sem filhos. Ele trabalhou na CEASA durante um ano, quando entrou tinha apenas dezenove anos e resolveu trabalhar lá porque precisava de dinheiro, estava desempregado e não tinha oportunidade no mercado formal. Ele não tinha experiência comprovada na carteira de trabalho e não possuia qualificação e capacitação que lhe pudesse proporcionar uma boa ocupação. Como a demanda por mão de obra era alta e a facilidade para conseguir um trabalho informal na CEASA é grande, alguns amigos o convidaram a ir trabalhar lá e ele acabou aceitando.

Cristiano descreve o trabalho que ele realizava na CEASA como um trabalho extremamente, pois ele precisava acordar cedo, carregar muito peso e ser extremamente explorado, além de receber pouco. Como Cristiano só trabalhava as terças e quintas só ganhava uma diária correspondente no valor de trinta e cinco reais (isso porque era no setor que vendia predominantemente morango, o preço da diária varia em cada local de trabalho), o horário de trabalho era de duas da manhã até ao meio-dia, no entanto eventualmente ele entrava as nove da noite e só saia ao meio-dia (mesmo assim continuava ganhando o mesmo valor da remuneração diária dos outros dias). A principal dificuldade que ele encontrou foi de se acostumar com o horário de trabalho.

No seu cotidiano de trabalho, Cristiano Martins desenvolvia diversas atividades, desde limpar o caminhão até organizar e vender a mercadoria comercializada, a atividade trabalhista se tornava pesada porque a atividade desenvolvida era descarregar as caixas com morango e levar para outros compartimentos onde a mercadoria era armazenada ou vendida, no entanto mesmo tendo como função principal descarregar a mercadoria, Cristiano nos diz que desenvolvia qualquer atividade que fosse pedida.

Com toda essa exploração e com tantas dificuldades um ponto positivo foi que Cristiano nunca sofreu nenhum acidente no trabalho, o que é algo frequente entre os trabalhadores informais da CEASA que trabalham descarregando os caminhões com frutas e outros alimentos. A seguir Cristiano nos relata que passou por uma situação complicada no trabalho:

\begin{citacao}
Teve um dia que eu estava organizando as mercadorias e não tinha nenhum carrinho para levar para os outros compartimentos, então o encarregado me pediu para levar uma caixa na cabeça que pesava aproximadamente uns cinquenta quilos, mas eu disse que não ia levar, até porque não tinha condição por causa do meu porte físico, eu achei um total absurdo.
\end{citacao}

Cristiano se sentia totalmente explorado no trabalho que realizava, principalmente porque ele trabalhava muito e ganhava pouco, não era respeitado, não era valorizado de nenhuma maneira. Mas como ele precisava muito do dinheiro, ia somente por isso, e com essa diária que ele ganhava custeava suas necessidades pessoais.

Na CEASA trabalham diversas pessoas, algumas com intenções de ganhar a dinheiro trabalhando honestamente e outros nem tanto, sabendo que é comum ocorrer alguns desvios e furto de mercadoria o entrevistado nos diz que aonde ele trabalhou acontecia sim esse tipo de coisa, mas que não são todos os trabalhadores que costumam fazer isso e também depende de cada lugar, cada local tem sua particularidade para esses desvios, se o dono, por exemplo, não controla sua mercadoria fica mais fácil desviar, mas se ele controla existe outras maneiras de burlar o controle do patrão.

Segundo o depoimento de Cristiano isso ocorria no seu setor de trabalho:

\begin{citacao}
Quando eu e o funcionário mais antigo fazíamos entrega no caminhão para outros locais fora da CEASA o funcionário mais antigo, que era o responsável por toda mercadoria, acabava desviando algumas caixas, porque o dono do morango não tinha um controle rigoroso da mercadoria e isso facilitava o desvio.
\end{citacao}

Cristiano apesar de não concordar com o desvio, também não gostava de opinar sobre isso, preferia fingir que aquilo não existia para não se prejudicar e ficar visto como delator entre os funcionários. Mesmo com essa adversidade, percebe-se que existem pessoas boas que trabalham informalmente porque precisam e não porque estão mal intencionadas. Apesar de que o desvio de carga é algo comum, não podemos generalizar e pensar que todos os trabalhadores da CEASA são mal intencionados.

Os dois próximos entrevistados da pesquisa ainda continuam trabalhando na CEASA. O primeiro deles é Alexandre Rabelo, pardo, vinte e um anos, solteiro, ensino médio incompleto. Antes de começar a trabalhar, Alexandre estava desempregado e somente estudava. Alexandre começou a trabalhar com dezenove anos e o que mais o motivou para ir trabalhar lá foi à falta de outras oportunidades de emprego e também a necessidade por um trabalho rápido.

Os seus amigos lhe motivaram a ir trabalhar na CEASA e ele acabou aceitando. Alexandre trabalha de segunda-feira até sábado, com dois horários diferentes. Nos dias de terça-feira e quinta-feira, ele entra às onze e meia da noite e sai às nove e meia da manhã. E nos outros dias que ele trabalha, o horário é de duas e meia da madrugada até às dez da manhã, seu ganho semanal é de duzentos e setenta e cinco reais. Sua única folga é no domingo e teve suas primeiras férias depois de um ano de trabalho.

O trabalho desenvolvido por Alexandre é o de separar as frutas boas das frutas ruins e ele considera boa sua atividade de trabalho , só acha ruim o fato de não ter tempo específico para merendar, é só merendar rapidamente e já voltar para o serviço.

Quando falta algum outro funcionário, isso lhe faz trabalhar mais para compensar a ausência do seu colega de trabalho e é nesse momento que ele se sente explorado no trabalho.

A única dificuldade encontrada por Alexandre foi se acostumar com o horário de trabalho e o trajeto casa-trabalho que lhe deixa inseguro devido ao fato de terem poucas pessoas no horário que ele sai para trabalhar, fora isso ele nos relata que o resto foi fácil.

Com relação à experiência adquirida no trabalho, ele nos diz que foi de extrema importância para sua vida, porque o tornou mais responsável e lhe ajudou a conquistar muitos dos seus objetivos pessoais, além de poder ajudar sua mãe financeiramente, que era uma motivação particular de Alexandre.

O segundo entrevistado é David Martins, pardo, dezoito anos, masculino, solteiro, ensino médio incompleto. Antes de começar a trabalhar na CEASA, era apenas estudante do ensino médio. Trabalha desde os dezesseis anos de idade. David optou por ir trabalhar na CEASA porque estava precisando ajudar a família e precisava de um trabalho rápido. Seus amigos indicaram a CEASA como um local onde facilmente se encontra um emprego porque lá sempre estão precisando de mão de obra.

Com a influência dos amigos, David conseguiu o trabalho e começou rapidamente. Já na segunda semana de trabalho estava adaptado ao serviço. Os dias de trabalho de David são as terças-feiras e quintas-feiras, com horário de entrada a uma hora da madrugada e horário de saída as dez horas da manhã, com uma remuneração diária de cinquenta reais, e cinco reais para custear sua merenda.

Segundo o relato de David seu trabalho é carregar e descarregar os caminhões com as caixas de frutas e fazer as entregas, dentre outras atividades que sejam necessárias para o melhor desenvolvimento do serviço. Sua principal dificuldade no começo foi acostumar-se com o horário de trabalho, mas com o passar dos anos isso se tornou algo normal e hoje não existe mais essa dificuldade.

Além do horário, David também passou por um momento complicado no começo, quando sofreu um acidente quando estava descarregando o caminhão com outro colega. Por um descuido, seu colega acabou jogando uma caixa na sua cabeça que pesa aproximadamente uns quarenta quilos, depois desse ocorrido seu patrão só perguntou se ele queria ir para casa e não ofereceu nenhuma ajuda para levá-lo ao médico ou sequer o indenizou por esse acidente sofrido no serviço. Passado essas dificuldades iniciais, hoje David se considera totalmente adaptado ao serviço.

David nos diz também que sua relação com os colegas de trabalho e com seu patrão é boa, o ambiente é sempre de descontração. Quando pergunto se ele se sente explorado no seu trabalho ele responde que sim e diz ``Não quero mais isso para mim e nem para ninguém, porque além do trabalho ser muito ruim, o pior de tudo foi que atrapalhou meus estudos e eu parei no primeiro grau do ensino médio''.

Com relação às boas coisas que o trabalho lhe trouxe, ele só consegue citar o dinheiro, que serve muito para ajudar sua família e custear seus interesses pessoais como comprar roupa e utilizar o dinheiro para sair com os amigos.

O que conseguimos notar em relação a esses trabalhadores da CEASA que foram entrevistados é que existe uma necessidade de ajudar a família como a si mesmo, então eles acabam aceitando o primeiro trabalho que aparece, sem se preocupar com as consequências futuras como, por exemplo, deixar os estudos para o segundo plano. Mas a própria pressão familiar por uma ajuda financeira os faz optar por esse caminho.

O fato é que a atividade de trabalho realizada é verdadeiramente o que menos importa para esses trabalhadores, eles só se preocupam com a remuneração, mesmo que tenham que sofrer qualquer tipo de adversidade por parte da atividade exercida.

Entrevistamos outra pessoa que trabalha informalmente conhecida popularmente como Sra. Dina , que antes de trabalhar informalmente era costureira. Ela possui uma barraca de churrasco que fica localizada no município de Maracanaú mais precisamente no bairro da Pajuçara. 

Dina como é popularmente conhecida tem cinquenta e sete anos de idade, divorciada, três filhos, é de cor parda, e só concluiu até a sétima série do ensino fundamental (fundamental incompleto). Ela nos relata um pouco da história de como tudo começou para hoje ela ser uma trabalhadora informal que consegue ganhar a vida e de certa maneira ter muitos benefícios com sua profissão. 

Tudo começou quando por acaso, o espetinho e o ponto já tinham um dono e esse dono era seu amigo, certo dia ele adoeceu e devido a sua doença ser um pouco grave, segundo Sra. Dina, ele ofereceu a barraca e o ponto para ela por cinco mil reais e disse que ela poderia pagar de qualquer forma. Ela aceitou a proposta e começou a tocar o investimento.

Ela passou um ano trabalhando praticamente só e depois de um ano de muita dificuldade contratou um funcionário para ajuda-la (ela paga cem reais por semana para ele), passado essas dificuldades hoje ela vende em média cento e cinquenta espetinhos por dia, fora os refrigerantes, cerveja, cachaça e baião nos dia de sexta-feira.

Ela não se queixa do trabalho, a única adversidade para ela é que uma parte do trabalho (comercialização das mercadorias) é realizada no meio da rua, mas ressalta a importância do investimento para sua vida e descreve com algo essencial para sua existência pessoal e profissional. 

Há pouco tempo Sra. Dina contratou outro funcionário com a mesma remuneração do seu primeiro funcionário e ela relata que sua relação com os funcionários é boa, a única cobrança é com relação a higiene deles. Ela libera seus funcionários do trabalho apenas no domingo e no final do ano dá vinte dias de férias no mês de dezembro. E já faz três anos que ela está desenvolvendo está atividade e se considera muito satisfeita e realizada por ter essa profissão.

No dia vinte e seis de junho de dois mil e quatorze realizamos a pesquisa etnográfica com Sra. Dina e seus dois funcionários. Chegamos à sua residência as sete e meia, horário em que ela e seus ajudantes começaram a cortar as carnes, fazer os temperos e preparar os espetos para serem comercializados mais tarde.

Tudo é realizado em um compartimento especifico dentro de sua própria casa, que é onde todas as preparações dos alimentos ocorrem e também onde as carnes são separadas e onde estão os vários utensílios que são usados no serviço. Antes de chegarmos Sra. Dina e seu marido já tinham ido até a CEASA as quatro horas da manhã comprar as carnes que serão preparadas para a comercialização.

Todo procedimento é feito por Dina e seus ajudantes, ela primeiramente lava toda a carne e depois começa a cortar toda carne é separa em varias vasilhas de plástico numa mesa separada apenas para realização desse serviço. Separando cada carne no seu devido lugar e seus ajudantes vão colocando as carnes nos palitos de espeto.

Com muita técnica eles colocam a carne no palito, mas mesmo com muito cuidado às vezes acabam furando os dedos em algum momento de descuido. Esse procedimento de corte e ajuste da carne acontece entre as sete e meia da manhã até as onze e meia da manhã. Depois disso, Sra. Dina e seus ajudantes vão almoçar. Os ajudantes almoçam em suas respectivas casas e depois retornam ao serviço.

Os ajudantes voltam ao serviço a uma e meia da tarde e começam a organizar os objetos que serão levados até o ponto de comércio, que localiza-se aproximadamente uns trinta a quarenta metros da casa de Dina. 

Ela começaorganizando as mesas e cadeiras, copos, vasilhas com os espetos crus, copos descartáveis, sacolas, balde de água que serve para lavar as mãos tanto dos ajudantes quanto dos clientes, a churrasqueira, o carvão e vários pedaços de madeira e papelão, uma lona que serve para proteção contra o sol e as demais matérias que serão comercializados entre eles cachaça e refrigerante.

Aproximadamente as duas horas da tarde os ajudantes saem para começar o trabalho no ponto de comércio e Sra. Dina fica em casa descansando ou fazendo outras atividades que julgue ser necessária. Quando chegam, os ajudantes primeiramente montam a barraca e colocam as mesas e cadeiras por perto, mas ainda não as deixa morganizadas, depois eles montam a churrasqueira, colocam um cesto de lixo no chão e o balde de água em uma mesa separada.

Ao término desse procedimento de ajustes os ajudantes colocam o carvão na churrasqueira e misturam com os pedaços de madeira e papelão para ascender o fogo. Eles têm um pouco de dificuldade para ascender o fogo por causa dos ventos.

Eles nos relatam que o ruim do trabalho é lidar com a fumaça e com a alta temperatura. Depois de ascender o fogo eles começam a assar as carnes de quatorze em quatorze espetinhos , que é o que cabe na churrasqueira de uma só vez, vão assando até terminarem tudo, esse procedimento dura em média duas horas ou duas horas e meia, enquanto isso a clientela é fraca, aparecem poucas pessoas, uns para beber uma dose de cachaça e outros para comer um espetinho.

Quando é aproximadamente umas cinco horas da tarde, Sra. Dina chega e pede para os ajudantes organizarem as mesas e cadeiras, pouco tempo depois a clientela começa a surgir em maior número. Mas neste dia, ela se descuida e é enganada por um cliente que pede o espeto, come, porém sai de sem pagar e isso lhe deixa bastante irritada, mas o serviço continua.

Os clientes chegam de vários locais e de diversos meios de condução, a pé, de carro, de motos, de bicicleta acompanhada por outras pessoas ou não. Quando o movimento aumenta, quem atendente quase toda a clientela são os ajudantes, Sra. Dina passa a maior parte do tempo conversando com alguns clientes ou amigos, uma vez ou outra ela atende algum cliente depois volta a ficar sentada conversando. 

Quando é aproximadamente umas seis e meia da noite, o movimento chega ao auge, todas as mesas estão lotadas e várias pessoas pedem os diversos produtos oferecidos, isso permanece até umas sete e meia da noite, depois desse horário aparecem outros clientes, mas o movimento diminui bastante, mas o trabalho continua até às oito e meia da noite.

Depois desse horário todo os materiais são recolhidos pelos ajudantes de Sra. Dina, primeiro eles começam a levar a cadeira, depois levam também as mesas, algumas são levadas em um carrinho de mão e outras são levadas nos braços de um dos ajudantes. Quando eles terminam essa tarefa, começam a organizar os materiais e guardam tudo em um compartimento dentro do carrinho de espeto, derramam a água suja que foi utilizada.

Quando tudo está organizado os dois ajudantes levam os carrinhos de volta para casa de Sra. Dina, que fica bem próximo, e colocam dentro da sua garagem (lugar em que ficam os dois carrinhos do espeto, carvão e a churrasqueira). Depois que tudo foi guardado, os espetinhos que sobram são colocados no freezer e segundo Dina ainda podem ser aproveitados até oito dias depois. 

É descarregada toda a mercadoria contida dentro do carrinho e colocada no compartimento da casa, onde ficam os materiais específicos para o trabalho, para mais tarde serem lavados por Dina e seu marido. Depois de descarregarem as mercadorias os ajudantes estão liberados do serviço e voltam para suas casas aproximadamente às nove horas da noite para recomeçar novamente a jornada de trabalho às sete e meia da manhã do outro dia.

Depois que os funcionários se vão, Dina vai contar o dinheiro que foi ganho no dia de trabalhao. Nesse dia a venda dos produtos rendeu cento e noventa e sete reais. Depois da contagem do dinheiro, nosso trabalho etnográfico já está encerrado, isso aproximadamente umas nove e meia da noite.