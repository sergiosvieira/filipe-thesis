\chapter{Definições e Conceitos Sobre o Trabalho Informal E Precarização}

Aqui abordaremos o trabalho informal sob a perspectiva de grandes intelectuais dentre eles Ricardo Antunes 
e Robert Castel para tentar deixar mais claro algumas questões relacionadas ao trabalho informal e a 
precariedade contida no mesmo. Esclarecendo uma duvida para nós a precariedade está inserida dentro do 
trabalho informal e no conceito da palavra trabalho informal está inserida muitas questões pertinentes que 
mostram a diversidade contida dentro de uma só palavra, que trata não apenas de pessoas que trabalham na 
informalidade de uma forma digamos honesta até pessoas que trabalham na informalidade de uma forma digamos 
criminosa. 

\section{O trabalho informal e a Precarização sob a perspectiva dos intelectuais}

É necessário entender que as relações existentes no trabalho informal não estão a margem do capital, ao 
contrário só existem porque o sistema permite que tal forma de organização seja produzida. Considerado 
como sinônimo de atraso e que seria eliminado com a evolução do capitalismo, o setor informal mostrou 
através de seus mecanismos mostrou que não é tão simples excluí-lo. 

O setor informal é de muita importância para o próprio capitalismo servindo até mesmo de proteção social.
Contrariando a teoria da subordinação sobre a crescente expansão do trabalho informal, fica explicito 
que o setor informal não se trata apenas de atividades que proporcionam um meio de sobrevivência. É 
relevante compreender que os estereótipos em volta do setor informal muitas vezes mascaram a realidade 
de um trabalho formal que talvez esteja na mesma proporção de desproteção social do setor informal.

\begin{citacao}
o capital necessita cada vez menos do trabalho estável e cada vez mais das mais 
diversificadas formas de trabalho parcial ou part-time, terceirizado, que são em 
escala crescente, parte constitutiva do processo de produção capitalista. \cite{antunes2011modos}
\end{citacao}

O trabalho como forma de inserção social é mostrado no relatório de Boisionat que é necessário que 
todos tenham um trabalho para não serem excluídos no sistema capitalista modernizado, além de ser 
muito importante para a inserção social. 

Através da flexibilidade do trabalho podemos notar que os defensores do capital assumem a tarefa 
de mascarar as contradições do capitalismo e sempre procuram realçar sua superfície para acharmos 
que é a sua essência. Por meio de suas formas de exploração é possível enganar a mão de obra 
assalariada lhe dando uma falsa autonomia em alguns empregos exercidos, que é marcada por um 
trabalho feito por resultados. 

Neste contexto de flexibilização do capital e de suas formas de 
trabalho fica cada vez mais difícil acreditar nas estatísticas acerca do mercado de trabalho 
porque está cada vez mais complicado identificar emprego e desemprego no contexto atual.

O desemprego na contemporaneidade se tornou algo complexo devido aos mecanismos que o próprio 
capitalismo permite como, por exemplo, o trabalho informal, nas suas mais variadas formas desde 
que não seja de estrita sobrevivência. Fica cada vez mais difícil identificar quem realmente é 
desempregado e quem não é, a verdade é que o que continua é a exploração do trabalho com varias 
estratégias que mascaram a realidade social em que vivemos e seja qual for à organização de 
trabalho no sistema capitalista não devemos esquecer que o lucro, ou seja, a mais-valia continua 
sendo o foco principal. 

Não se pode esquecer que a mais-valia é o que sustenta e controla todo 
o capitalismo, ou seja, é a maior quantidade de trabalho não pago se transformando em lucro. 
Portanto, ``Não se deve esquecer que a premissa de o produtor conter o máximo possível de trabalho 
não pago só pode ser alterada para mais'' e também que ``se existe uma mão invisível que rege a 
produção capitalista, está é sem duvida, a lei do valor''.

Segundo \cite{antunes1999sentidos} a sociedade do trabalho abstrato possibilitou, por meio da constituição 
de uma massa de trabalhadores, a aparência de sociedade fundada no descentramento da categoria 
trabalho e na perda de centralidade do ato laborativo no mundo contemporâneo. Nesse contexto fica 
exposto, o fato de que o entendimento das mutações em curso no mundo operário nos obriga ir além 
das aparências, ao fazer isso acredita-se que o sentido dado ao trabalho pelo capital é completamente 
diverso do sentido atribuído pela humanidade.

\begin{citacao}
A Lógica do sistema produtor de mercadorias vem convertendo a concorrência e a busca da produtividade 
num processo destrutivo que tem gerado uma imensa precarização do trabalho e aumento monumental do 
exército industrial de reserva, do número de desempregados. \cite{antunes1999sentidos} 
\end{citacao}

Neste breve trecho Ricardo Antunes \cite{antunes1999sentidos} nos alerta que a concorrência é um grande vilão para a 
grande massa de trabalhadores que, vendem sua força de trabalho, porque essa concorrência transforma 
a produtividade em um processo destrutivo e com isso quem vai sofrer as piores consequências serão 
os operários que serão cada vez mais explorados, e assim o capitalista busca vencer a concorrência 
que existe entre empresas ou empregadores dentro do processo existente no sistema capitalista, 
fazendo os donos do capital enriquecer cada vez mais.

Por consequência quando um ou alguns trabalhadores procuram reivindicar suas condições precárias 
de trabalho, por serem explorados excessivamente de todas as formas possíveis dentro do seu ambiente 
de trabalho, acabam em sua maioria sendo demito e consequentemente aumentando os indicies de indivíduos 
que estão desempregados por conta de um sistema que necessita que existam desempregados em grande 
massa para poder explorar cada vez mais quem só tem sua força de trabalho para vender em troca de uma 
remuneração minúscula. 

Então a partir desse fenômeno que permite que muitos fiquem desempregados, alguns buscam outro meio de 
sobrevivência e de tentar realizar seus objetivos individuais. Isso faz com que esses trabalhadores que 
estão desempregados e não conseguem retornar ao trabalho consolidado pelas leis de trabalho, busquem 
uma nova atividade econômica para viver. 

Sabendo que muitos trabalho de carteira assinada são extremamente 
precários, que sempre busca desvalorizar a grande massa de trabalhadores que se sacrificam para fazer 
a economia da empresa fluir e não são minimamente valorizados, muitos acabam migrando para outra atividade 
trabalhista (trabalho informal) que, diga-se de passagem, é muito rentável para alguns no Brasil, mesmo 
que seja tão ou até mais precarizado do que o trabalho formal (Depende do ponto de vista de cada um). 

Essa grande massa busca no trabalho informal uma melhoria na condição de vida, principalmente porque 
muitas atividades do trabalho informal são de baixos investimentos e antes quem era subordinado agora 
é quem manda. E isso em um sistema opressor como o que estamos inseridos faz uma enorme diferença. 
Portanto, essas pessoas que foram de certa forma excluída pelo trabalho formal estão migrando para o 
setor de trabalho e economias informais estão buscando além de tudo condições mínimas de dignidade 
humana, capital econômico, autonomia de certa forma, mesmo que essa autonomia não seja algo tão 
perfeito como essas pessoas imaginam etc.

Para a grande maioria dos brasileiros (senso comum) o trabalho informal está sempre associado e chamado 
popularmente de trabalho sem carteira assinada, não que essas pessoas estejam completamente erradas, 
mas o trabalho informal tem mais particularidades, porém até em pesquisas realizadas por entidades de 
pesquisa do governo brasileiro o trabalho informal é sim definido como um trabalho sem carteira.

Segundo \cite{noronha2003informal} a palavra informalidade é bastante ampla e tem diversos conceitos subentendidos, 
então ele afirma que seria mais interessante deixar de falar ou escrever a palavra informalidade no 
tema do trabalho informal e passarmos a entender e falar em ``contratos atípicos'', porque segundo sua 
visão seria mais contextual para o tema especifico e de melhor entendimento de todos que fossem 
pesquisar, debater ou até mesmo ler o algo relacionado com o trabalho informal, principalmente quando 
estamos falando de contratos de trabalho no Brasil.

É preciso ser bastante cauteloso quando falamos de trabalho informal no contexto brasileiro até porque 
cada país detém suas particularidades com relação ao tema abordado, então não devemos cometer o equivoco 
de achar que sabemos todos os conceitos e definições da palavra trabalho informal, existem muitas definições 
e regras especificas para cada contexto então, por exemplo, aqui no Brasil algo que é considerado informal 
pouco tempo depois pode ser considerado formal ou legal, isso porque o capitalismo está em constante mudança 
e sempre é preciso adaptar novas formas de trabalho que se encaixem no sistema capitalista para que se possa 
cobrar mais carga tributária. 

Como o capitalismo está em constante mudança é bastante complicado compreender 
e destrinchar sua lógica estrutural mais especificadamente, não a lógica que é repassada por meio de uma 
aparência que busca mascarar seu sistema estrutural, no entanto é preciso entender que mesmo que sejam entendidos 
e compreendidos muitos conceitos inseridos no trabalho e na economia informal é necessário ter consciência de 
que não existe uma verdade única, até porque no contexto de humanidades ,sabemos que as ciências sócias não
são feitas de verdades absolutas e incontestáveis e consequentemente não podem prever o futuro que o trabalho 
informal ira percorrer ao longo dos próximos anos no Brasil, só podemos supor.

O sistema capitalista, com toda sua estrutura permite o que chamamos de trabalho informal ou até mesmo o que 
Noronha chama de “contratos atípicos” e dentro desses trabalhos informais ou desses tipos de contratos, existe 
uma grande massa de trabalhadores e trabalhadoras que vivem desse trabalho considerado um trabalho precarizado. 
Esses trabalhadores que só possuem sua força de trabalho como meio de inclusão socioeconômica e como não possuem
muita qualificação escolar acabam optando por algo considerado mais fácil ou até mesmo uma saída para ganhar dinheiro. 

Iremos adentrar dentro do setor informal para procurar entender melhor o trabalho informal no Brasil, com suas
particularidades e dificuldades para esses que estão nesse contexto. O Brasil com sua imensa e diversificada
população é considerado um país de economia periférica, porque sua grande maioria populacional não é dotada de 
tanto capital econômico, porque como sabemos o capital econômico no caso brasileiro é concentrado na mão de 
poucos e com eles estão às decisões sobre o mercado e também como a grande massa de trabalhadores são visto 
nesse mercado de fluxo desregulado. 

Sendo assim é preciso deixar explicito que através de nossas próprias 
políticas e através do próprio sistema capitalista foi criado um sistema precarizado de postos de trabalho, 
que desvaloriza cada vez mais o cidadão e busca somente explora-lo demasiadamente, só se importando como fazer 
a mão de obra ficar mais e mais barateada e inventar e reutilizar mecanismos passados (que deram certo) para 
intensificar a exploração e a busca por mais-valia. 

Através disso nos deparamos com um grande problema que é 
a falta de oportunidades para todos pelo menos tentarem não se tornar mão de obra barata e serem obrigados a 
vender sua força de trabalho para serem explorados tanto no ambiente de trabalho como fora dele.

Sabemos que a desigualdade entre as pessoas inseridas na lógica do mercado de trabalho brasileiro não existe 
apenas em nosso país, mas precisamos olhar para nós mesmos e buscarmos métodos para diminuir esse problema 
estrutural com ações eficazes para não deixar cada vez mais precarizado nossos trabalhadores que estão em 
ocupações que ficam na parte inferior da pirâmide do mercado econômico brasileiro, mas são de extrema 
importância para o desenvolvimento do sistema econômico do país. 

Falar de trabalho é sempre complexo, porque respiramos essa palavra quase que em cem por cento de nossa vida, 
seja numa conversa entre amigos, seja quando escolhemos ingressar numa faculdade, curso profissionalizante, 
curso técnico, ou até mesmo quando estamos tentando conquistar alguém, por isso é preciso tomar cuidado quando 
falamos de trabalho precário até porque estamos falando de pessoas que mesmo sem entenderem que realizam essas 
atividades estão envolvidas nesse meio e é preciso respeita-las sem fazer juízo de valor.

Sabemos que o trabalho em nosso é país fortemente marcado por uma estrutura forte de desigualdade entre as 
atividades e entre as pessoas que realizam essas atividades, dificilmente vamos mudar essa desigualdade entre 
o detentor da mão de obra e o que vende sua força de trabalho, mas quero deixar claro que não é apenas no 
trabalho formal que existe desigualdade, essa desigualdade também existe no trabalho informal até porque não 
é porque uma atividade é informal não irão existir relações hierarquicamente desiguais e injustas.

Como já sabemos a desigualdade seja ela em qualquer contexto social é muito presente em nosso país, porque o 
capitalismo existe não para proteger e lutar pela classe operária e sim para fazer as pessoas acreditar na 
utopia de que somos todos iguais e com as mesmas oportunidades. 

Então o sistema capitalista utiliza a classe 
de trabalhadores menos favorecidas para aumentar o lucro do mercado econômico brasileiro, e assim ser visto 
no exterior como um pais que está em constante desenvolvimento e capaz de fazer parte das melhores e mais 
bem conceituadas entidades externas e assim fazerem bons acordos com países centrais. No entanto a verdade 
é que o mais importante é somente explorar e conseguir obter o máximo de mais-valia possível de cada trabalhador. 

Então como o sistema capitalista nos mostra que não existem oportunidades para todos e nossa grande maioria 
aceita esse dogma nem todos são considerados pessoas (no sentido sociologicamente atribuído a palavra) e nem 
todos conseguirão realizar seus sonhos por mais simples que sejam. Por isso é necessário buscar e encontrar 
algumas formas de sobrevivência e não apenas de sobrevivência como também encontrar uma forma de se tornar 
pessoa com todos os atributos que enxergamos em alguém que é considerado pessoa na lógica do nosso sistema. 
Então o trabalho informal surge como forma de fazerem aqueles que foram deixados de lado pelo trabalho 
formal continuarem lutando por seus objetivos tanto pessoais como profissionais.

\begin{citacao}
Em sociedades democráticas a lei é, por definição, justa. Caso não seja, deve ser mudada, mas nunca desprezada. 
Contudo, muitos contratos considerados justos por determinados grupos não são previstos em lei ou são francamente
ilegais. Além disso, no Brasil, popularmente, o trabalho ``informal'' típico pode ser entendido, se não como ``justo'',
ao menos como ``aceitável'', e certamente não é considerado ``ilegal'' a menos que se trate de crime (em geral comércio
de produtos ilegais) e não apenas um contrato ilícito. \cite{noronha2003informal}
\end{citacao}

No Brasil o trabalho informal está diretamente ligado à atividade as margens da lei tanto por não existir leis 
especificam que abordem o trabalho informal, como também por existir atividades consideradas criminosas dentro 
do trabalho informal, no entanto é preciso enfatizar que o trabalho informal pode estar à margem da lei, mas não 
a margem do sistema capitalista, porque o próprio sistema permite a existência dessas atividades. 

No trabalho formal, por exemplo, pode e existem atividades informais como na contratação de trabalhadores sem carteira assinada. 
Então é bastante complexo essa relação entre forma e informal, legal e ilegal. O próprio sistema que te oprime e
que não da uma proteção e uma oportunidade de vida melhor, permite uma meio que se torne uma saída que muda a vida
de muitos, porém não de todos, nem de todos que estão inseridos no trabalho e na economia informal, como nem de 
todos que são excluídos pelo trabalho formal e tentaram o trabalho informal como beneficio pessoal e familiar. 

Cada um que busca uma oportunidade nesse sistema vai encontrar diversas dificuldades, algumas parecidas com as
mesmas do trabalho formal, algumas distintas. Mas o mundo do trabalho informal com certeza não é para todos é 
preciso de uma série de atributos que vão desde uma boa argumentação e persuasão até encontrar uma atividade 
informal rentável que lhe proporcione uma melhoria significativa de vida.

Então é necessário mostrar, que mesmo com as imensas dificuldades existentes por cada um que trabalha informalmente
o trabalho informal é considerado de certa forma um meio de proteção social para estes trabalhadores que estão
nesta atividade. Porque através dessa atividade informal de trabalho o indivíduo consegue retornar a ter um lugar
na economia do país e conseguem ter as mínimas condições de dignidade para alguém que está inserido no sistema 
capitalista do Brasil.

O fato é que mesmo que seja considerado como marginalizado e precário o trabalho informal é tido como algo bom 
para sociedade brasileira que está inserida nessa atividade e que necessita de uma ocupação para dar continuidade
em sua vida pessoal e profissional buscando e realizando seus objetivos mesmo que estejam esquecidos pela lei.
Mas não tão bom para o país que deixa de arrecadar impostos dessas atividades, por não regulamentar essas 
atividades que tem um imenso impacto na economia geral do Brasil.

No Brasil o trabalho informal está relacionado as atividades que não possuem uma regulamentação governamental, 
mas mesmo assim é relatada sua existência em alguns órgãos de pesquisa como no IBGE. Mesmo que existam diversas 
atividades no mundo do trabalho informal brasileiro todas essas atividades são consideradas ilegais para as leis 
do Brasil (Por falta de legislação especifica), porém sabemos que hoje no contexto atual uma atividade que é 
considerada ilegal possa mudar de contexto e passaremos a ter uma legislação e uma regulamentação que beneficie 
muitas atividades que são informais, porém não são ilegais. 

Existem sim muitas atividades informais que são ilegais no sentido de serem criminosas, mas também existem diversas atividades que são informais, mas não são
criminosas e são essas atividades que devem ser regulamentadas para poder mudar esse contexto do trabalho 
informal brasileiro que é tido sempre como uma atividade marginalizada e precarizada. 

Para melhorar a situação
de todos que precisam do trabalho informal para sobreviver, ajudar a família, complementar a renda, se sentirem
pessoas uteis para sociedade, possuírem capital econômico, realização dos objetivos e consequentemente melhorar
o mercado brasileiro e sua economia. Porém para que esses benéficos sejam conquistados é preciso políticas públicas
eficazes para que se resolva essa temática.

Com sua imensa complexidade o trabalho e a economia informal são estudados e  considerados como algo ruim para 
a economia brasileira(para alguns intelectuais), porém não podemos generalizar e esquecer que existem pessoas
que estão sendo beneficiadas as margens da lei, mas não as margens do opressor sistema capitalista. O sistema 
permite a existência da atividade informal, mas é prejudicado por não conseguir regulamentar esse e isso é tido
como um problema estrutural que dificilmente vai mudar por completo devido a grande dimensão de atividades 
existentes dentro do trabalho informal.

Cada um que trabalha informalmente tem diversas dificuldades dentre as quais podemos citar: excesso de horas 
trabalhadas, dificuldade para controlar as finanças (no caso de quem é investidor), não tiram férias remunerada,
não existem folga remunerada, não existe como já sabemos legislação que procure proteger o trabalhador, não podem 
ficar doente, muitas vezes falta de apoio na atividade exercida, nenhuma estabilidade etc.

Outro problema gerado por falta de uma regulação do mercado no trabalho informal no Brasil é o fato de que as 
estatísticas podem e devem estar erradas em muitas pesquisas, porque se em nosso país carteira assinada é sinônimo 
de estar empregado e sem carteira assinada é não estar empregado, fica subentendido que não existe outra forma de 
emprego se não o trabalho formal, porém na realidade não é bem assim que acontece. Então todos esses trabalhadores 
que se empregam em atividades informais são considerados desempregados mesmo que estejam trabalhando e consumindo
diariamente. Por esse e outros fatores é sempre necessário levar em consideração o trabalho informal no Brasil 
para se entender melhor várias questões dentre elas o desemprego estrutural.

Partindo deste pensamento de Maria Augusta Tavares ``Não é o operário que usa os meios de produção é o meio de produção
que usa o operário'' Podemos perceber como o capitalismo se comporta quando se trata de mão de obra de trabalho, 
percebe-se através dessa simples frase uma realidade bem clara quando estudamos o trabalho seja formal ou informal.

Em um país com oportunidades diferentes para seus habitantes a classe subalterna sempre fica esquecida quando é preciso
que alguém seja beneficiado, por isso esses que são esquecidos por um país com estruturas que são difíceis de serem 
mudadas, tentam de alguma maneira ir atrás seus objetivos, mesmo com todas as dificuldades encontradas por quem está
na classe inferior. 

No contexto estudado a classe operaria sempre é quem fica com os piores empregos, com a pior 
educação, com os piores serviços essenciais para uma vida digna, porém mesmo com todas essas adversidades essa categoria
de trabalhadores é uma categoria guerreira que luta diariamente por uma condição de ganhar uma melhor remuneração sem
se preocupar com a precarização do trabalho exercido e enfrentando de frente esse problema que é a precariedade do 
trabalho informal. 

Mas com tanta desvalorização tanto como seres humanos que precisam ter direitos básicos e também 
como vendedores de força de trabalho, o proletariado brasileiro não fica apenas esperando por politicas sociais eficazes,
eles correm atrás de melhorias de vida através de uma atividade realizada com inúmeras adversidades, mas que se tornou 
o único meio para muitos de sobreviver e lutar por uma condição melhor.

O trabalho informal e sua economia não são importantes para as pessoas que não precisam dessa atividade; porém para 
quem está dentro dessa ocupação ele é tido como algo bom tanto sociologicamente falando como economicamente. Fazendo
muitas pessoas encontrarem uma nova ocupação e também um novo meio de interação social que é fundamental na vida de 
qualquer ser humano, servindo de proteção para os menos favorecidos e uma forma de mostrar que o sistema capitalista 
é tão complexo que ele permite que o proletariado possa se auto-ajudar e ao mesmo tempo de alguma forma prejudicar o
desenvolvimento econômico do país. 

\begin{citacao}
O salário reconhece e remunera o trabalho ``em geral'', isto é, atividades potencialmente úteis para todos. Assim na
sociedade contemporânea, e para a maioria de seus membros é o fundamento de sua cidadania econômica. Também está no
princípio da cidadania social: esse trabalho representa a participação de cada um numa sociedade. É assim o ponto 
médio concreto sobre o qual se constroem direitos e deveres sociais, responsabilidades e reconhecimento, ao mesmo 
tempo que sujeições e coerções. \cite{castel1998metamorfoses}
\end{citacao}

Além de tudo o que já foi destacado é importante salientar que em qualquer sociedade existe algo que é de extrema 
importância, no caso, o salário, porque dentro do salário está inserida diversas questões sociais que ajudam cada 
individuo a entender melhor sua vida, porém é necessário que essa importância seja notada por parte do trabalhador 
subalterno para que não troque sua força de trabalho por uma remuneração que não revela verdadeiramente o sofrimento 
e a dedicação contida no seu trabalho.

Através do direito do trabalho se conseguiu alguns benefícios para os trabalhadores, mesmo que com algumas 
fragilidades (no caso a falta de fiscalização por parte do Estado Social).O direito trabalhista melhorou muito
a vida de muitos trabalhadores principalmente em relação as demissões, fez com que a relação entre empregado e 
empregador se equilibrasse porque se tornou necessário um motivo sério para demitir.

Segundo \cite{castel1998metamorfoses} O desemprego estrutural nos mostra a importância que o Estado Social tem para toda nação,
porque regula e protege a sociedade, porém faz isso segundo seus interesses e é preciso entender esses interesses
para essa proteção não ficar mascarada e não ter verdadeiro sentido na prática, então o empregado pode estar 
totalmente fora da seguridade social se o Estado não se fizer presente.

\begin{citacao}
Não para banalizar a gravidade do desemprego. Contudo, enfatizar essa precarização do trabalho permite 
compreender os processos que alimentam a vulnerabilidade social e produzem, no final do percurso, o desemprego 
e a desfiliação. \cite{castel1998metamorfoses}
\end{citacao}

