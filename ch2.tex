\chapter{Definições e Conceitos Sobre o Trabalho Informal E Precarização}

Aqui abordaremos o trabalho informal sob a perspectiva de grandes intelectuais, dentre eles Ricardo Antunes e Robert Castel, para tentar deixar mais claro algumas questões relacionadas ao trabalho informal e a precariedade contida no mesmo. 

A precariedade está inserida dentro do trabalho informal e em seu conceito estão inseridas muitas questões pertinentes que mostram a diversidade contida dentro dela, que trata não apenas de pessoas que trabalham na informalidade de uma forma honesta ou mesmo criminosa.

Primeiramente faz-se necessário entender que as relações existentes no trabalho informal não estão à margem do sistema capitalista, pelo contrário só existe porque o sistema permite que tal forma de organização seja produzida e reproduzida diariamente. 

O setor informal, já considerado como sinônimo de atraso e que seria eliminado com a evolução do capitalismo, mostrou-se através de seus mecanismos que ele não seria facilmente excluído, mensurado ou controlado na sua totalidade. Ele é de muita importância para o próprio capitalismo servindo até mesmo de proteção social. 

Contrariando a teoria da subordinação \footnote{Teoria formulada em 1980, segundo a qual o setor informal é uma forma de produção subordinada e intersticial à produção capitalista. Nessa visão, o espaço econômico onde o setor informal atua é destruído, criado e recriado pelo movimento da acumulação capitalista. Paulo Renato C. Souza, Salário e emprego em economias atrasadas.Campinas: Unicamp/ IE, 1999} sobre a crescente expansão do trabalho informal, fica explicito que o setor informal não se trata apenas de atividades que proporcionam um meio de sobrevivência.

\begin{citacao}
o capital necessita cada vez menos do trabalho estável e cada vez mais das mais 
diversificadas formas de trabalho parcial ou part-time, terceirizado, que são em 
escala crescente, parte constitutiva do processo de produção capitalista. \cite{antunes2011modos}
\end{citacao}

O trabalho como forma de inserção social é mostrado no relatório de Boisionat \footnote{Em 1995, o Relatório Boissonat, concluiu que, no horizonte de vinte anos o emprego continuaria sendo um meio essencial de inserção social. Portanto, se a tecnologia economiza trabalho, é melhor desdobrar os empregos existentes para que todos tenha um, do que dá-los a uns e privar permanentemente outros.} que diz que é necessário que todos tenham um trabalho para não serem excluídos no sistema capitalista modernizado, além de ser muito importante para a inserção social.

Através da flexibilização do trabalho \footnote{A flexibilização do trabalho basicamente é uma nova condição para que a mão de obra consiga um emprego nos dias atuais, porque é necessário que se desempenhe mais de uma função dentro da empresa para não ser subsitituido por outra pessoa que aceite desempenhar mais de uma função, é necessário ser flexível e aceitar desempenhar várias
funções dentro da empresa.} podemos notar que os defensores do capital assumem a tarefa de mascarar as contradições do capitalismo e sempre procuram realçar sua superfície para acharmos que é a sua essência. Por meio de suas formas de exploração é possível enganar a mão de obra assalariada lhe dando uma falsa autonomia em alguns empregos exercidos, que é marcada por um trabalho feito
por resultados.

Neste contexto de flexibilização das formas de trabalho, fica cada vez mais difícil acreditar nas estatísticas acerca do mercado de trabalho porque está cada vez mais complicado identificar emprego e desemprego no contexto atual. 

O desemprego na contemporaneidade se tornou algo complexo devido aos mecanismos que o próprio capitalismo permite como, por exemplo, o trabalho informal, nas suas mais variadas formas desde que não seja de estrita sobrevivência. 

Fica cada vez mais difícil identificar quem realmente é desempregado e quem não é, a verdade é que o que continua é a exploração do trabalho com varias estratégias que mascaram a realidade social em que vivemos e seja qual for à organização de trabalho no sistema capitalista não devemos esquecer que o lucro, ou seja, a mais-valia continua sendo o foco principal.

\begin{citacao}
A lógica do sistema produtor de mercadorias vem convertendo a concorrência e a busca da produtividade num processo destrutivo que tem gerado uma imensa precarização do trabalho e aumento monumental do exército industrial de reserva, do número de desempregados. (ANTUNES, pág 18, 2009) 
\end{citacao}

\cite{antunes2009infoproletarios} nos alerta que a concorrência é um grande vilão para a
grande massa de trabalhadores que, vendem sua força de trabalho, porque essa
concorrência transforma a produtividade em um processo destrutivo e com isso
quem vai sofrer as piores consequências serão os operários que serão cada vez
mais explorados, e assim o capitalista busca vencer a concorrência que existe
entre empresas ou empregadores dentro do processo existente no sistema
capitalista, fazendo os donos do capital enriquecer cada vez mais.

%Nossa missão é mostrar a precarização dentro do trabalho informal, iremos dar mais ênfase nessa questão socialmente relevante para o entendimento e conhecimento de todos.

Por conseqüência, quando um ou alguns trabalhadores procuram reivindicar melhores condições de trabalho, algumas vezes, eles acabam sendo demitidos, aumentando o índice de desemprego, contribuindo ainda mais para o sistema capitalista, que necessita que existam muitos desempregados para poder explorar cada vez mais quem só tem sua força de trabalho para vender.

Então a partir dessa característica do capitalismo que permite que muitos fiquem desempregados, alguns buscam outro meio de sobrevivência para realizar seus objetivos individuais. Isso faz com que esses trabalhadores que estão desempregados e não conseguem retornar ao trabalho formal, ou seja, com a verdadeira proteção social e estatal regida principalmente pelas CLT (Consolidação das Leis Trabalhistas), busquem uma atividade econômica informal para viver. 

Para a grande maioria dos brasileiros, o trabalho informal está associado ao ``sem carteira assinada''. Até em pesquisas realizadas por entidades de pesquisa do governo brasileiro, o trabalho informal é sim definido como um trabalho sem carteira. Não que essas pessoas estejam completamente erradas, mas o trabalho informal tem mais particularidades.

Muitos trabalhos com carteira assinada são extremamente precários, a grande maiooria dos postos de trabalhos existentes em nosso país não valoriza o trabalhador assalariado que dedica sua energia e empenho para contribuir com o lucro da empresa. Então muitos acabam migrando para o trabalho informal na expectativa de ser mais bem remunerados.

A grande massa busca no trabalho informal uma melhoria na condição de vida, principalmente porque muitas atividades dele são de baixos investimentos. Essas pessoas buscam condições mínimas de dignidade humana, rendimento e autonomia, mesmo que essa autonomia não seja total.

Segundo \cite{noronha2003informal} a palavra informalidade é bastante ampla e tem diversos conceitos subentendidos, então ele afirma que seria mais interessante deixar de falar ou escrever a palavra informalidade no tema do trabalho informal e passarmos a entender e falar em contratos atípicos, porque segundo sua visão seria mais contextual para o tema específico e de melhor entendimento de todos que fossem pesquisar, debater ou até mesmo ler algo relacionado com o trabalho informal, principalmente quando estamos falando de contratos de trabalho no Brasil.

É preciso ser bastante cauteloso quando falamos de trabalho informal no contexto brasileiro até porque cada país detém suas particularidades com relação ao tema abordado, então não devemos cometer o equivoco de achar que sabemos todos os conceitos e definições da palavra trabalho informal, existem muitas definições e regras especificas para cada contexto então, por exemplo, aqui no Brasil algo que é considerado informal pouco tempo depois pode ser considerado formal ou legal, isso porque o capitalismo está em constante mudança e sempre é preciso adaptar novas formas de trabalho que se encaixem no sistema para que se possa cobrar mais carga tributária.

Como o capitalismo está em constante mudança é bastante complicado compreender e destrinchar sua lógica estrutural mais especificadamente, não a lógica que é repassada por meio de uma aparência que busca mascarar seu sistema estrutural, no entanto é preciso entender que mesmo que sejam entendidos e compreendidos muitos conceitos inseridos no trabalho e na economia informal é necessário ter consciência de que não existe uma verdade única, até porque no contexto de humanidades, sabemos que as ciências sócias não são feitas de verdades absolutas e incontestáveis e conseqüentemente não podem prever o futuro que o trabalho informal irá percorrer ao longo dos próximos anos no Brasil, só podemos supor.

O sistema capitalista, com toda sua estrutura, permite denominarmos esse tipo de trabalho como informal ou até mesmo contratos atípicos \cite{noronha2003informal}. Neste tipo de trabalho existe uma grande massa de trabalhadores que vivem desse trabalho considerado precarizado. Existe sim uma forte relação entre trabalho informal e precarização do ponto de vista que os trabalhadores informais não possuem direitos trabalhistas fundamentais, como férias remuneradas, décimo terceiro salário, licença maternidade etc.

O trabalho informal é considerado precarizado devido as adversidades contidas dentro dele, que o transforma numa atividade trabalhísticas complexa do ponto de vista das condições mínimas que possa garantir a integridade física e moral do trabalhador.

Esses trabalhadores sem qualificação específica ou maior escolaridade acabam optando por algo considerado mais fácil ou até mesmo uma saída para ganhar alguma remuneração.

Sendo assim é preciso deixar explicito que através de nossas próprias políticas e através do próprio sistema capitalista foi criado um sistema precarizado de postos de trabalho, que desvaloriza cada vez mais o cidadão e busca somente explorá-lo demasiadamente, só se importando como fazer a mão de obra ficar mais e mais barata e inventar e reutilizar mecanismos passados (que deram certo) para intensificar a exploração e a busca por mais-valia.

Sabemos que a desigualdade entre as pessoas inseridas na lógica do mercado de trabalho brasileiro não existe apenas em nosso país, mas precisamos olhar para nós mesmos e buscarmos métodos para diminuir esse problema estrutural com ações eficazes para não deixar cada vez mais precarizado nossos trabalhadores que estão em ocupações que ficam na parte inferior da pirâmide do mercado econômico brasileiro, mas são de extrema importância para o desenvolvimento do sistema econômico do país.

Falar de trabalho é sempre complexo, porque respiramos essa palavra constantemente na nossa vida, seja numa conversa entre amigos, seja quando escolhemos ingressar numa faculdade, curso profissionalizante, curso técnico, ou até mesmo quando estamos tentando conquistar alguém, por isso é preciso tomar cuidado quando falamos de trabalho precário até porque estamos falando de pessoas que mesmo sem entenderem que realizam essas atividades estão envolvidas nesse meio e é preciso respeitá-las sem fazer juízo de valor.
 
O sistema capitalista utiliza a classe de trabalhadores menos favorecidas para aumentar o lucro do mercado econômico brasileiro, e assim ser visto no exterior como um país que está em constante desenvolvimento e é capaz de fazer parte das melhores e mais bem conceituadas entidades externas e assim fazerem bons acordos com países centrais. No entanto a verdade é que o mais importante é somente explorar e conseguir obter o máximo de mais-valia possível de cada trabalhador.

Então como o sistema capitalista nos mostra que não existem oportunidades para todos e nossa grande maioria aceita esse dogma nem todos são considerados pessoas, (no sentido sociologicamente atribuído a palavra) e nem todos conseguirão realizar seus sonhos por mais simples que sejam. Por isso é necessário buscar e encontrar algumas formas de viver dignamente com  todos os atributos que enxergamos em alguém que é considerado pessoa na lógica do nosso sistema. Então o trabalho informal surge como forma de fazer com que aqueles que foram deixados de lado pelo trabalho formal continuarem lutando por seus objetivos tanto pessoais como profissionais.

\begin{citacao}
Em sociedades democráticas a lei é, por definição, justa. Caso não seja, deve ser mudada, mas nunca desprezada. Contudo, muitos contratos considerados justos por determinados grupos não são previstos em lei ou são francamente ilegais. Além disso, no Brasil, popularmente, o trabalho ``informal'' típico pode ser entendido, se não como ``justo'', ao menos como ``aceitável'', e certamente não é considerado ``ilegal'' a menos que se trate de crime (em geral comércio de produtos ilegais) e não apenas um contrato ilícito. (NORONHA, pág. 121, 2003)
\end{citacao}

No Brasil, o trabalho informal está diretamente ligado à atividade as margens da lei, como também por existir atividades consideradas criminosas dentro do trabalho informal, no entanto é preciso enfatizar que o trabalho informal não está a margem do sistema capitalista, porque o próprio sistema permite a existência dessa atividade.

Então é bastante complexa essa relação entre formal e informal, legal e ilegal. O próprio sistema que oprime e que muitas vezes não da uma proteção e uma oportunidade de vida melhor, permite uma saída que muda a vida de muitos, porém não de todos.


Cada um que busca uma oportunidade nesse sistema vai encontrar diversas dificuldades, algumas parecidas com as mesmas do trabalho formal, algumas distintas. Mas o mundo do trabalho informal com certeza não é para todos é preciso de uma série de atributos que vão desde uma boa argumentação e persuasão até encontrar uma atividade rentável que lhe proporcione uma melhoria significativa de vida. 

Então é necessário mostrar, que mesmo com as imensas dificuldades existentes, o trabalho informal é considerado de certa forma um meio de proteção social. Porque através dele o indivíduo consegue retornar a ter um lugar na economia do país e consegue ter as mínimas condições de dignidade.

O fato é que mesmo que seja considerado como marginalizado e precário, o trabalho informal é tido como algo bom para sociedade brasileira. Mas não tão bom para o país que deixa de arrecadar impostos dessas atividades que tem um imenso impacto na economia geral do Brasil.

No Brasil o trabalho informal está relacionado às atividades que não possuem uma regulamentação governamental, mas mesmo assim é relatada sua existência em alguns órgãos de pesquisa como no IBGE (Instituto Brasileiro de Geografia e Estatística). 

Mesmo que existam diversas atividades, elas são consideradas ilegais para as leis do Brasil (Por falta de legislação especifica), porém sabemos que hoje no contexto atual, uma atividade considerada ilegal pode passar a legalidade. Existem sim muitas atividades informais que são ilegais no sentido de serem criminosas, mas também existem diversas atividades que são informais, mas não são criminosas e são essas atividades que devem ser regulamentadas para poder mudar esse contexto do trabalho informal brasileiro que é tido sempre como uma atividade marginalizada e precarizada.


Cada um que trabalha informalmente tem diversas dificuldades dentre as quais podemos citar: excesso de horas trabalhadas, dificuldade para controlar as finanças (no caso de quem é investidor), não tiram férias remunerada, não existe folga remunerada, não ``podem'' ficar doente, muitas vezes falta de apoio até mesmo da própria família na atividade exercida, nenhuma estabilidade etc.

Outro problema existente dentro do trabalho informal no Brasil é o fato de que as estatísticas podem estar erradas em muitas pesquisas, porque se em nosso país carteira assinada é sinônimo de estar empregado, fica subentendido que não existe outra forma de emprego se não o trabalho formal, porém na realidade não é bem assim que acontece. 
Todos esses trabalhadores que se empregam em atividades informais são considerados desempregados mesmo que estejam trabalhando e consumindo diariamente. Por esse e outros fatores é sempre necessário levar em consideração o trabalho informal no Brasil para se entender melhor várias questões, dentre elas o desemprego estrutural. 

Para \cite{augusta} ``Não é o operário que usa os meios de produção é o meio de produção que usa o operário''. Podemos perceber através dessa frase como o capitalismo se comporta quando se trata da mão de obra assalariada, porque devido a grande importância de possuir uma ocupação trabalhista o operário aceita desempenhar qualquer função para poder continuar dentro do mercado de trabalho.

Em um país com oportunidades diferentes para seus habitantes a classe subalterna sempre fica esquecida quando é preciso que alguém seja beneficiado, por isso, esses que são esquecidos por um país com estruturas que são difíceis de serem mudadas, tentam de alguma maneira ir atrás seus objetivos, mesmo com todas as dificuldades encontradas por quem é considerado de uma classe inferior. 

No contexto estudado, a classe operária sempre é quem fica com os piores empregos, com a pior educação, com os piores serviços essenciais para uma vida digna, porém mesmo com todas essas adversidades, essa categoria de trabalhadores consegue sobreviver buscando sempre melhorar sua realidade.

Mas com tanta desvalorização, tanto como seres humanos que precisam ter direitos básicos, quanto como vendedores de força de trabalho, o proletariado brasileiro não fica apenas esperando por políticas sociais eficazes, eles buscam melhorias de vida através de uma atividade realizada com inúmeras adversidades, mas que se tornou o único meio para muitos de sobreviver e lutar por uma condição melhor.

O trabalho informal e sua economia não são importantes para as pessoas que não precisam dessa atividade; porém para quem está dentro dessa ocupação ele é tido como algo bom, tanto sociologicamente, quanto economicamente. 

Fazendo muitas pessoas encontrarem uma nova ocupação e também um novo meio de interação social que é fundamental na vida de qualquer ser humano, servindo de proteção para os menos favorecidos e uma forma de mostrar que o sistema capitalista é tão complexo que ele permite que o proletariado possa se auto-ajudar e ao mesmo tempo prejudicar o desenvolvimento econômico do país.

\begin{citacao}
O salário reconhece e remunera o trabalho ``em geral'', isto é, atividades potencialmente úteis para todos. Assim na sociedade contemporânea, e para a maioria de seus membros é o fundamento de sua cidadania econômica. Também está no princípio da cidadania social: esse trabalho representa a participação de cada um numa sociedade. É assim o ponto médio concreto sobre o qual se constroem direitos e deveres sociais, responsabilidades e reconhecimento, ao mesmo tempo que sujeições e coerções. (CASTEL, pág 581, 2010)
\end{citacao}

Além de tudo o que já foi destacado é importante salientar que em qualquer sociedade existe algo que é de extrema importância, o salário, porque dentro dele estão inserida diversas questões sociais que ajudam o individuo a entender melhor sua vida, porém é necessário que essa importância seja notada por parte do trabalhador subalterno para que ele não troque sua força de trabalho por uma remuneração subvalorizada, porém sabemos que na prática isso é muito complicado de acontecer.
